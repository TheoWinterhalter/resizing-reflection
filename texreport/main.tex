\documentclass[a4paper,english]{lipics-utf8x}

\usepackage[T1]{fontenc} %

\usepackage{amsmath, amssymb, amsfonts, stmaryrd}
\usepackage{pifont}
\PrerenderUnicode{é} % For the author names in the heading

% Add some colors
\usepackage[usenames,dvipsnames,svgnames,table]{xcolor}
\usepackage{hyperref}
\hypersetup{
 linktocpage,
 colorlinks,
 citecolor=BlueViolet,
 filecolor=red,
 linkcolor=Blue,
 urlcolor=BrickRed
}

\usepackage{graphicx}
\usepackage{placeins}

% Meta comment
\newcommand\meta[1]{\noindent\textcolor{blue}{\emph{#1}}}

% Include the macro file
% evergreens
\newcommand{\der}{\,\vdash}
\newcommand{\Der}{\,\Vdash}

% semantic brackets
\def\lv{\mathopen{{[\kern-0.14em[}}}    % opening [[ value delimiter
\def\rv{\mathclose{{]\kern-0.14em]}}}   % closing ]] value delimiter
\newcommand{\den}[1]{\lv #1 \rv}
\newcommand{\Den}[3][]{\den{#2}^{#1}_{#3}}
\newcommand{\dent}[2]{\llparenthesis#1\rrparenthesis_{#2}}

% latin etc. abbrev
\newcommand{\abbrev}[1]{#1} % alternative: \emph{#1}
\newcommand{\cf}{\abbrev{cf.}\ }
\newcommand{\eg}{\abbrev{e.\,g.}}
\newcommand{\Eg}{\abbrev{E.\,g.}}
\newcommand{\ie}{\abbrev{i.\,e.}}
\newcommand{\Ie}{\abbrev{I.\,e.}}
\newcommand{\etal}{\abbrev{et.\,al.}}
\newcommand{\wwlog}{w.\,l.\,o.\,g.} % \wlog is ``write into log file''
\newcommand{\Wlog}{W.\,l.\,o.\,g.}
\newcommand{\wrt}{w.\,r.\,t.}

% space-preserving paragraph headings (for lipics)
\newcommand{\subheading}[1]{\subparagraph{#1.}} %Alt: \subsection{#1}
\newcommand{\paradot}[1]{\subparagraph{#1.}}

% Inference rules
\newcommand{\rulename}[1]{\ensuremath{\mbox{\sc#1}}}
\newcommand{\ru}[2]{\dfrac{\begin{array}[b]{@{}c@{}} #1 \end{array}}{#2}}
\newcommand{\rux}[3]{\ru{#1}{#2}~#3}
\newcommand{\nru}[3]{#1\ \ru{#2}{#3}}
\newcommand{\nrux}[4]{#1\ \ru{#2}{#3}\ #4}
\newcommand{\dstack}[2]{\begin{array}[b]{c}#1\\#2\end{array}}
\newcommand{\dru}[3]{\ru{\dstack{#1}{#2}}{#3}}
\newcommand{\tru}[4]{\dru{\dstack{#1}{#2}}{#3}{#4}}
\newcommand{\trux}[5]{\dru{\dstack{#1}{#2}}{#3}{#4}\ #5}
\newcommand{\qru}[5]{\tru{\dstack{#1}{#2}}{#3}{#4}{#5}}
\newcommand{\ndru}[4]{#1\ \ru{\dstack{#2}{#3}}{#4}}
\newcommand{\ndrux}[5]{#1\ \ru{\dstack{#2}{#3}}{#4}\ #5}

% Symbols and names
\newcommand\Type{\operatorname{Type}}
\newcommand\isnType[2]{\operatorname{is-}#1\operatorname{-Type}\ #2}
\newcommand\nType[1]{#1\operatorname{-Type}}
\newcommand\R{\operatorname{R}}
\newcommand\emb[2]{\operatorname{embedding}#1\ #2}
\newcommand\RRe[2]{\operatorname{RR_e}#1\ #2}
% \newcommand\type{\ \bm{\operatorname{type}}}
\DeclareMathOperator{\type}{\ \mathbf{type}}
\DeclareMathOperator{\ctr}{\mathbf{ctr}}
\DeclareMathOperator{\refl}{\mathbf{refl}}
\newcommand\rew{\searrow}
\newcommand\gettype{\operatorname{.type}}
\newcommand\getproof{\operatorname{.proof}}
\newcommand\Var{\operatorname{Var}}
\newcommand\Exp{\operatorname{Exp}}
\newcommand\Ctx{\operatorname{Ctx}}
\newcommand\Whnf{\operatorname{Whnf}}
\newcommand\Wne{\operatorname{Wne}}
\newcommand\hiff{\mathrel{\hat{\iff}}}
\newcommand\red{{\downarrow}}


% Title and so...
\title{Trust me I'm a an hProp}
\author[1]{Théo Winterhalter}

\begin{document}

  \maketitle

  \begin{abstract}
    This is an attempt to add one resizing rule that would subsume several of
    them, including the equivalence and the hProp ones.
  \end{abstract}

  \section{Syntax}

  \[
    \begin{array}{l@{~}l@{~}l@{~}r@{~}l@{\quad}l}
      \Var  & \ni & x,y,X,Y \\
      \Sort & \ni & s             & ::= & \Type_k \mbox{ }
                                                (k \in \mathbb{N}) \\
      \Exp  & \ni & t,u,T,U & ::= & s \mid \Pi x:U.T \mid \Sigma x:U.T \mid
                                    t =_T u \mid \RRe{U}{T}{h} \\
                         &&& \mid & x \mid \lambda x:U.t \mid t~u
                               \mid (t;u) \mid t.1 \mid t.2 \mid \refl_t \\
      \Ctx  & \ni & \Gamma  & ::= & \cdot \mid \Gamma, x:T \\
    \end{array}
  \]

  \noindent %
  We write $A \to B$ as short for $\Pi \_:A.B$, the non-dependent product.
  We also write $=$ for $=_T$ when $T$ is understood.
  %
  The notation $\emb{A}{B}$ denotes the type of embeddings of $A$ into $B$.
  \[
    \begin{array}{l@{\quad}l@{~}l@{~}l}
      & \emb{A}{B} &:=& \Sigma (f : A \to B). \isEquiv \ap_f \\
      \text{where} & \ap_f &:& (x =_A y) \to (f(x) =_B f(y)) \\
      \text{and} & \isEquiv f &:=& \Sigma_{(g : B \to A)}
                               \Sigma_{(\eta : g \circ f \sim \id_A)}
                               \Sigma_{(\varepsilon : f \circ g \sim \id_B)}
                               \Pi_{(x : A)} f~(\eta~x) = \varepsilon~(f~x) \\
      \text{and} & h_1 \sim h_2 &:=& \Pi_{(x : T)} h_1~x = h_2~x.
    \end{array}
  \]

  \section{Typing Rules}

  \begin{center}
  \(
    \ru{}{\der \cdot}
    \qquad
    \ru{\Gamma \der T : s \qquad
        x \notin \Gamma
      }{\der \Gamma, x : T}
    \qquad
    \ru{\der \Gamma
      }{\Gamma \der \Type_i : \Type_{i+1}}
    \qquad
    \ru{\der \Gamma \qquad
        (x : T) \in \Gamma
      }{\Gamma \der x : T}
  \)
  \end{center}

  \begin{center}
  \(
    \ru{\Gamma \der t : \Pi x:A.B \qquad
        \Gamma \der t' : A
      }{\Gamma \der t\ t' : B[t'/x]}
    \qquad
    \ru{\Gamma \der A : s \qquad
        \Gamma, x:A \der B : s' \qquad
        (s,s',s'') \in \R
      }{\Gamma \der \Pi x:A.B : s''}
  \)
  \end{center}

  \begin{center}
  \(
    \ru{\Gamma \der \Pi x:A.B : s \qquad
        \Gamma, x:A \der t : B
      }{\Gamma \der \lambda x:A.t : \Pi x:A.B}
    \qquad
    \ru{\Gamma \der t : \Sigma x:A.B
      }{\Gamma \der t.1 : A}
    \qquad
    \ru{\Gamma \der t : \Sigma x:A.B
      }{\Gamma \der t.2 : B[t.1/x]}
  \)
  \end{center}

  \begin{center}
  \(
    \ru{\Gamma \der A : s \qquad
        \Gamma, x:A \der B : s' \qquad
        (s,s',s'') \in \R
      }{\Gamma \der \Sigma x:A.B : s''}
    \qquad
    \ru{\Gamma \der T : s \qquad
        \Gamma \der t,t' : T
      }{\Gamma \der t =_T t' : s}
  \)
  \end{center}

  \begin{center}
  \(
    \ru{\Gamma \der t : A \qquad
        \Gamma \der t' : B[t/x] \qquad
        \Gamma \der \Sigma x:A.B : s
      }{\Gamma \der (t;t') : \Sigma x:A.B}
    \qquad
    \ru{\Gamma \der t : T
      }{\Gamma \der \refl_t : t =_T t}
  \)
  \end{center}

  \begin{center}
  \(
    \ru{\Gamma \der t : A \qquad
        \Gamma \der B : s \qquad
        \Gamma \der A \le B
      }{\Gamma \der t : B}
  \)
  \end{center}

  \noindent %
  Rules regarding resized types mirror those for the type they are
  \emph{hiding}.

  \begin{center}
  \(
    \ru{\Gamma \der A : \Type_i \qquad
        \Gamma \der B : \Type_j \qquad
        \Gamma \der h : \emb{A}{B}
      }{\Gamma \der \RRe{A}{B}{h} : \Type_j}
  \)
  \end{center}

  \begin{center}
  \(
    \ru{\Gamma \der t : \RRe{(\Pi x:A.B)}{C}{h} \qquad
        \Gamma \der t' : A
      }{\Gamma \der t\ t' : B[t'/x]}
  \)
  \end{center}

  \begin{center}
  \(
    \ru{\Gamma \der \RRe{(\Pi x:A.B)}{C}{h} : s \qquad
        \Gamma, x:A \der t : B
      }{\Gamma \der \lambda x:A.t : \RRe{(\Pi x:A.B)}{C}{h}}
  \)
  \end{center}

  \begin{center}
  \(
    \ru{\Gamma \der t : \RRe{(\Sigma x:A.B)}{C}{h}
      }{\Gamma \der t.1 : A}
    \qquad
    \ru{\Gamma \der t : \RRe{(\Sigma x:A.B)}{C}{h}
      }{\Gamma \der t.2 : B[t.1/x]}
  \)
  \end{center}

  \begin{center}
  \(
    \ru{\Gamma \der t : A \qquad
        \Gamma \der t' : B[t/x] \qquad
        \Gamma \der \RRe{(\Sigma x:A.B)}{C}{h} : s
      }{\Gamma \der (t;t') : \RRe{(\Sigma x:A.B)}{C}{h}}
  \)
  \end{center}

  \begin{center}
  \(
    \ru{\Gamma \der t : T \qquad
        \Gamma \der \RRe{(t =_T t)}{U}{h} : s
      }{\Gamma \der \refl_t : \RRe{(t =_T t)}{U}{h}}
  \)
  \end{center}

\section{Equality Rules}

\section{Cumulativity}

\section{Algorithmic Equality}

\paradot{Weak head normalization}

Weak head normal forms (whnfs) are given by the following grammar:

\begin{align*}
  \Whnf &\ni a,f,A,B,F &::=~& \Pi x:U.T \mid \Sigma x:U.T \mid t =_T u
                         \mid \RRe{U}{T}{t} \\
      &&\mid~& n \mid \lambda x:U.t \mid (t;u) \mid \refl_t \\
  \Wne  &\ni n,N &::=~& x \mid n~u \mid n.1 \mid n.2
\end{align*}

We present weak-head reduction as follows:

\begin{center}
\(
  \ru{t \rew f \qquad
      f~u \rew a
    }{t~u \rew a}
  \qquad
  \ru{}{a \rew a}
  \qquad
  \ru{t[u/x] \rew a
    }{(\lambda x:U.t)~u \rew a}
  \qquad
  \ru{}{n~u \rew n~u}
\)
\end{center}

\begin{center}
\(
  \ru{t \rew (u;v) \qquad
      u \rew a
    }{t.1 \rew a}
  \qquad
  \ru{t \rew (u;v) \qquad
      v \rew a
    }{t.2 \rew a}
  \qquad
  \ru{t \rew n
    }{t.1 \rew n.1}
  \qquad
  \ru{t \rew n
    }{t.2 \rew n.2}
\)
\end{center}

\end{document}
