\documentclass[a4paper,english]{lipics-utf8x}

\usepackage[T1]{fontenc} %

\usepackage{amsmath, amssymb, amsfonts, stmaryrd}
\usepackage{pifont}
\PrerenderUnicode{é} % For the author names in the heading

% Add some colors
\usepackage[usenames,dvipsnames,svgnames,table]{xcolor}
\usepackage{hyperref}
\hypersetup{
 linktocpage,
 colorlinks,
 citecolor=BlueViolet,
 filecolor=red,
 linkcolor=Blue,
 urlcolor=BrickRed
}

\usepackage{graphicx}
\usepackage{placeins}

% Meta comment
\newcommand\meta[1]{\noindent\textcolor{blue}{\emph{#1}}}

% Include the macro file
% evergreens
\newcommand{\der}{\,\vdash}
\newcommand{\Der}{\,\Vdash}

% semantic brackets
\def\lv{\mathopen{{[\kern-0.14em[}}}    % opening [[ value delimiter
\def\rv{\mathclose{{]\kern-0.14em]}}}   % closing ]] value delimiter
\newcommand{\den}[1]{\lv #1 \rv}
\newcommand{\Den}[3][]{\den{#2}^{#1}_{#3}}
\newcommand{\dent}[2]{\llparenthesis#1\rrparenthesis_{#2}}

% latin etc. abbrev
\newcommand{\abbrev}[1]{#1} % alternative: \emph{#1}
\newcommand{\cf}{\abbrev{cf.}\ }
\newcommand{\eg}{\abbrev{e.\,g.}}
\newcommand{\Eg}{\abbrev{E.\,g.}}
\newcommand{\ie}{\abbrev{i.\,e.}}
\newcommand{\Ie}{\abbrev{I.\,e.}}
\newcommand{\etal}{\abbrev{et.\,al.}}
\newcommand{\wwlog}{w.\,l.\,o.\,g.} % \wlog is ``write into log file''
\newcommand{\Wlog}{W.\,l.\,o.\,g.}
\newcommand{\wrt}{w.\,r.\,t.}

% Inference rules
\newcommand{\rulename}[1]{\ensuremath{\mbox{\sc#1}}}
\newcommand{\ru}[2]{\dfrac{\begin{array}[b]{@{}c@{}} #1 \end{array}}{#2}}
\newcommand{\rux}[3]{\ru{#1}{#2}~#3}
\newcommand{\nru}[3]{#1\ \ru{#2}{#3}}
\newcommand{\nrux}[4]{#1\ \ru{#2}{#3}\ #4}
\newcommand{\dstack}[2]{\begin{array}[b]{c}#1\\#2\end{array}}
\newcommand{\dru}[3]{\ru{\dstack{#1}{#2}}{#3}}
\newcommand{\tru}[4]{\dru{\dstack{#1}{#2}}{#3}{#4}}
\newcommand{\trux}[5]{\dru{\dstack{#1}{#2}}{#3}{#4}\ #5}
\newcommand{\qru}[5]{\tru{\dstack{#1}{#2}}{#3}{#4}{#5}}
\newcommand{\ndru}[4]{#1\ \ru{\dstack{#2}{#3}}{#4}}
\newcommand{\ndrux}[5]{#1\ \ru{\dstack{#2}{#3}}{#4}\ #5}

% Symbols and names
\newcommand\Type{\operatorname{Type}}
\newcommand\isnType[2]{\operatorname{is-}#1\operatorname{-Type}\ #2}
\newcommand\nType[1]{#1\operatorname{-Type}}
\newcommand\R{\operatorname{R}}
\newcommand\emb[2]{\operatorname{embedding}#1\ #2}
\newcommand\RRe[2]{\operatorname{RR_e}#1\ #2}
% \newcommand\type{\ \bm{\operatorname{type}}}
\DeclareMathOperator{\type}{\ \mathbf{type}}
\DeclareMathOperator{\ctr}{\mathbf{ctr}}
\DeclareMathOperator{\refl}{\mathbf{refl}}
\newcommand\rew{\searrow}
\newcommand\gettype{\operatorname{.type}}
\newcommand\getproof{\operatorname{.proof}}
\newcommand\Var{\operatorname{Var}}
\newcommand\Exp{\operatorname{Exp}}
\newcommand\Ctx{\operatorname{Ctx}}
\newcommand\Whnf{\operatorname{Whnf}}
\newcommand\Wne{\operatorname{Wne}}


% Title and so...
\title{Trust me I'm a an hProp}
\author[1]{Théo Winterhalter}

\begin{document}

  \maketitle

  \begin{abstract}
    This is an attempt to add one resizing rule that would subsume several of
    them, including the equivalence and the hProp ones.
  \end{abstract}

  \section{Syntax}

  \[
    \begin{array}{l@{~}l@{~}l@{~}r@{~}l@{\quad}l}
      \Var  & \ni & x,y,X,Y \\
      \Sort & \ni & s             & ::= & \Type_k \mbox{ }
                                                (k \in \mathbb{N}) \\
      \Exp  & \ni & t,u,T,U & ::= & s \mid \Pi x:U.T \mid \Sigma x:U.T \mid
                                    t =_T u \mid \RRe{U}{T}{h} \\
                         &&& \mid & x \mid \lambda x:U.t \mid t~u
                               \mid (t;u) \mid t.1 \mid t.2 \mid \refl_t \mid
                               \J (T,U,t_{refl},u_1,u_2,t_{eq}) \\
      \Ctx  & \ni & \Gamma  & ::= & \cdot \mid \Gamma, x:T \\
    \end{array}
  \]

  \noindent %
  We write $A \to B$ as short for $\Pi \_:A.B$, the non-dependent product.
  We also write $=$ for $=_T$ when $T$ is understood.
  %
  The notation $\emb{A}{B}$ denotes the type of embeddings of $A$ into $B$.
  \[
    \begin{array}{l@{\quad}l@{~}l@{~}l}
      & \emb{A}{B} &:=& \Sigma (f : A \to B). \isEquiv \ap_f \\
      \text{where} & \ap_f &:& (x =_A y) \to (f(x) =_B f(y)) \\
      \text{and} & \isEquiv f &:=& \Sigma_{(g : B \to A)}
                               \Sigma_{(\eta : g \circ f \sim \id_A)}
                               \Sigma_{(\varepsilon : f \circ g \sim \id_B)}
                               \Pi_{(x : A)} f~(\eta~x) = \varepsilon~(f~x) \\
      \text{and} & h_1 \sim h_2 &:=& \Pi_{(x : T)} h_1~x = h_2~x.
    \end{array}
  \]

  \section{Typing Rules}

  \begin{center}
  \(
    \ru{}{\der \cdot}
    \qquad
    \ru{\Gamma \der T : s \qquad
        x \notin \Gamma
      }{\der \Gamma, x : T}
    \qquad
    \ru{\der \Gamma
      }{\Gamma \der \Type_i : \Type_{i+1}}
    \qquad
    \ru{\der \Gamma \qquad
        (x : T) \in \Gamma
      }{\Gamma \der x : T}
  \)
  \end{center}

  \begin{center}
  \(
    \ru{\Gamma \der t : \Pi x:A.B \qquad
        \Gamma \der t' : A
      }{\Gamma \der t\ t' : B[t'/x]}
    \qquad
    \ru{\Gamma \der A : s \qquad
        \Gamma, x:A \der B : s' \qquad
        (s,s',s'') \in \R
      }{\Gamma \der \Pi x:A.B : s''}
  \)
  \end{center}

  \begin{center}
  \(
    \ru{\Gamma \der \Pi x:A.B : s \qquad
        \Gamma, x:A \der t : B
      }{\Gamma \der \lambda x:A.t : \Pi x:A.B}
    \qquad
    \ru{\Gamma \der t : \Sigma x:A.B
      }{\Gamma \der t.1 : A}
    \qquad
    \ru{\Gamma \der t : \Sigma x:A.B
      }{\Gamma \der t.2 : B[t.1/x]}
  \)
  \end{center}

  \begin{center}
  \(
    \ru{\Gamma \der A : s \qquad
        \Gamma, x:A \der B : s' \qquad
        (s,s',s'') \in \R
      }{\Gamma \der \Sigma x:A.B : s''}
    \qquad
    \ru{\Gamma \der T : s \qquad
        \Gamma \der t,t' : T
      }{\Gamma \der t =_T t' : s}
  \)
  \end{center}

  \begin{center}
  \(
    \ru{\Gamma \der t : A \qquad
        \Gamma \der t' : B[t/x] \qquad
        \Gamma \der \Sigma x:A.B : s
      }{\Gamma \der (t;t') : \Sigma x:A.B}
    \qquad
    \ru{\Gamma \der t : T
      }{\Gamma \der \refl_t : t =_T t}
  \)
  \end{center}

  \begin{center}
  \(
    \tru{\Gamma \der A : s \qquad
         \Gamma \der C : \Pi x:A. \Pi y:A. (x =_A y) \to s'
       }{\Gamma \der b : \Pi x:A. C\ x\ x\ \refl_x
       }{\Gamma \der u, v : A \qquad
         \Gamma \der p : u =_A v
       }{\Gamma \der \J (A,C,b,u,v,p) : C\ u\ v\ p}
  \)
  \end{center}

  \begin{center}
  \(
    \ru{\Gamma \der t : A \qquad
        \Gamma \der B : s \qquad
        \Gamma \der A \le B
      }{\Gamma \der t : B}
  \)
  \end{center}

  \noindent %
  Rules regarding resized types mirror those for the type they are
  \emph{hiding}.

  \begin{center}
  \(
    \ru{\Gamma \der A : \Type_i \qquad
        \Gamma \der B : \Type_j \qquad
        \Gamma \der h : \emb{A}{B}
      }{\Gamma \der \RRe{A}{B}{h} : \Type_j}
  \)
  \end{center}

  \begin{center}
  \(
    \ru{\Gamma \der t : \RRe{(\Pi x:A.B)}{C}{h} \qquad
        \Gamma \der t' : A
      }{\Gamma \der t\ t' : B[t'/x]}
  \)
  \end{center}

  \begin{center}
  \(
    \ru{\Gamma \der t : \Pi x:A.B \qquad
        \Gamma \der t' : \RRe{A}{C}{h}
      }{\Gamma \der t\ t' : B[t'/x]}
  \)
  \end{center}

  \meta{If we do that, then isn't the substitution ill-typed?}

  \begin{center}
  \(
    \ru{\Gamma \der \RRe{(\Pi x:A.B)}{C}{h} : s \qquad
        \Gamma, x:A \der t : B
      }{\Gamma \der \lambda x:A.t : \RRe{(\Pi x:A.B)}{C}{h}}
  \)
  \end{center}

  \begin{center}
  \(
    \ru{\Gamma \der t : \RRe{(\Sigma x:A.B)}{C}{h}
      }{\Gamma \der t.1 : A}
    \qquad
    \ru{\Gamma \der t : \RRe{(\Sigma x:A.B)}{C}{h}
      }{\Gamma \der t.2 : B[t.1/x]}
  \)
  \end{center}

  \begin{center}
  \(
    \ru{\Gamma \der t : A \qquad
        \Gamma \der t' : B[t/x] \qquad
        \Gamma \der \RRe{(\Sigma x:A.B)}{C}{h} : s
      }{\Gamma \der (t;t') : \RRe{(\Sigma x:A.B)}{C}{h}}
  \)
  \end{center}

  \begin{center}
  \(
    \tru{\Gamma \der A : s \qquad
         \Gamma \der C : \Pi x:A. \Pi y:A. (x =_A y) \to s'
       }{\Gamma \der b : \Pi x:A. C\ x\ x\ \refl_x
       }{\Gamma \der u, v : A \qquad
         \Gamma \der p : \RRe{(u =_A v)}{B}{h}
       }{\Gamma \der \J (A,C,b,u,v,p) : C\ u\ v\ p}
  \)
  \end{center}

  \meta{This raises the problem of how to give a $\RRe{A}{B}{h}$ when $A$ is
  required.}

  \begin{center}
  \(
    \ru{\Gamma \der t : T \qquad
        \Gamma \der \RRe{(t =_T t)}{U}{h} : s
      }{\Gamma \der \refl_t : \RRe{(t =_T t)}{U}{h}}
  \)
  \end{center}

  \section{Equality Rules}

  \paradot{Computation ($\beta$) and extensionality ($\eta$)}

  \begin{center}
  \(
    \ru{\Gamma, x:U \der t:V \qquad
        \Gamma \der u : U
      }{\Gamma \der (\lambda x:U.t)~u = t[u/x] : V[u/x]}
    \qquad
    \ru{\Gamma \der t : \Pi x:U.V
      }{\Gamma \der t = \lambda x:U.t~x : \Pi x:U.V}
  \)
  \end{center}

  \begin{center}
  \(
    \ru{\Gamma \der t : U \qquad
        \Gamma \der t' : V[t/x]
      }{\Gamma \der (t;t').1 = t : U}
    \qquad
    \ru{\Gamma \der t : U \qquad
        \Gamma \der t' : V[t/x]
      }{\Gamma \der (t;t').2 = t' : V[t/x]}
  \)
  \end{center}

  \begin{center}
  \(
    \ru{\Gamma \der t : \Sigma x:U.T
      }{\Gamma \der t = (t.1 ; t.2) : \Sigma x:U.T}
  \)
  \end{center}

  \begin{center}
  \(
    \dru{\Gamma \der A : s \qquad
         \Gamma \der C : \Pi x:A. \Pi y:A. (x =_A y) \to s'
       }{\Gamma \der b : \Pi x:A. C\ x\ x\ \refl_x \qquad
         \Gamma \der u : A
       }{\Gamma \der \J (A,C,b,u,u,\refl_u) = b\ u : C\ u\ u\ \refl_u}
  \)
  \end{center}

  \paradot{Equivalence Rules}

  \begin{center}
  \(
    \ru{\Gamma \der t : T
      }{\Gamma \der t = t : T}
    \qquad
    \ru{\Gamma \der t' = t : T
      }{\Gamma \der t = t' : T}
    \qquad
    \ru{\Gamma \der t_1 = t_2 : T \qquad
        \Gamma \der t_2 = t_3 : T
      }{\Gamma \der t_1 = t_3 : T}
  \)
  \end{center}

  \paradot{Compatibility Rules}

  \begin{center}
  \(
    \rux{\Gamma \der U = U' : s \qquad
         \Gamma, x:U \der V = V' : s'
       }{\Gamma \der \Pi x:U.V = \Pi x:U'.V' : s''
       }{(s,s',s'') \in \R}
  \)
  \end{center}

  \begin{center}
  \(
    \ru{\Gamma \der U = U' : s \qquad
        \Gamma, x:U \der V : s' \qquad
        \Gamma, x:U \der t = t' : V
      }{\Gamma \der \lambda x:U.t = \lambda x:U'.t' : \Pi x:U.V}
  \)
  \end{center}

  \begin{center}
  \(
    \ru{\Gamma \der t = t' : \Pi x:U.V \qquad
        \Gamma \der u = u' : U
      }{\Gamma \der t~u = t'~u' : V[u/x]}
  \)
  \end{center}

  \begin{center}
  \(
    \rux{\Gamma \der U = U' : s \qquad
         \Gamma, x:U \der V = V' : s'
       }{\Gamma \der \Sigma x:U.V = \Sigma x:U'.V' : s''
       }{(s,s',s'') \in \R}
  \)
  \end{center}

  \begin{center}
  \(
    \ru{\Gamma \der t_1 = t'_1 : U \qquad
        \Gamma, x:U \der V : s \qquad
        \Gamma \der t_2 = t'_2 : V[t_1/x]
      }{\Gamma \der (t_1;t_2) = (t'_1;t'_2) : \Sigma x:U.V}
  \)
  \end{center}

  \begin{center}
  \(
    \ru{\Gamma \der t = t' : \Sigma x:U.V
      }{\Gamma \der t.1 = t'.1 : U}
    \qquad
    \ru{\Gamma \der t = t' : \Sigma x:U.V
      }{\Gamma \der t.2 = t'.2 : V[t.1/x]}
  \)
  \end{center}

  \begin{center}
  \(
    \ru{\Gamma \der T = T' \qquad
        \Gamma \der t = t' : T \qquad
        \Gamma \der u = u' : T
      }{\Gamma \der t =_T u = t' =_{T'} u'}
  \)
  \end{center}

  \begin{center}
  \(
    \ru{\Gamma \der t = t' : T \qquad
      }{\Gamma \der \refl_t = \refl_{t'} : t =_T t}
  \)
  \end{center}

  \begin{center}
  \(
    \tru{\Gamma \der A = A' : s \qquad
         \Gamma \der C = C' : \Pi x:A. \Pi y:A. (x =_A y) \to s'
       }{\Gamma \der b = b' : \Pi x:A. C\ x\ x\ \refl_x
       }{\Gamma \der u = u' : A \qquad
         \Gamma \der v = v' : A \qquad
         \Gamma \der p = p' : u =_A v
       }{\Gamma \der \J (A,C,b,u,v,p) = \J (A',C',b',u',v',p') : C\ u\ v\ p}
  \)
  \end{center}

  \begin{center}
  \(
    \ru{\Gamma \der U = U' : s \qquad
        \Gamma \der T = T' : s' \qquad
        \Gamma \der h, h' : \emb{U}{T}
      }{\Gamma \der \RRe{U}{T}{h} = \RRe{U'}{T'}{h'} : s'}
  \)
  \end{center}

  \paradot{Conversion Rule}

  \begin{center}
  \(
    \ru{\Gamma \der t = t' : T \quad
        \Gamma \der T \le T'
      }{\Gamma \der t = t' : T'}
  \)
  \end{center}


  \section{Cumulativity}

  \begin{center}
  \(
    \rux{}{\Gamma \der \Type_i \le \Type_j}{i \le j}
    \qquad
    \ru{\Gamma \der U = U' : s \qquad
        \Gamma, x:U \der T \le T'
      }{\Gamma \der \Pi x:U.T \le \Pi x:U'.T'}
  \)
  \end{center}

  \begin{center}
  \(
    \ru{\Gamma \der T = T' : s
      }{\Gamma \der T \le T'}
    \qquad
    \ru{\Gamma \der T_1 \le T_2 \qquad
        \Gamma \der T_2 \le T_3
      }{\Gamma \der T_1 \le T_3}
  \)
  \end{center}

  \section{Algorithmic Equality}

  \paradot{Weak head normalization}

  Weak head normal forms (whnfs) are given by the following grammar:

  \begin{align*}
    \Whnf &\ni a,f,A,B,F &::=~& \Pi x:U.T \mid \Sigma x:U.T \mid t =_T u
                           \mid \RRe{U}{T}{t} \\
        &&\mid~& n \mid \lambda x:U.t \mid (t;u) \mid \refl_t \\
    \Wne  &\ni n,N &::=~& x \mid n~u \mid n.1 \mid n.2
  \end{align*}
  %
  We present weak-head reduction as follows:

  \begin{center}
  \(
    \ru{t \rew f \qquad
        f~u \rew a
      }{t~u \rew a}
    \qquad
    \ru{}{a \rew a}
    \qquad
    \ru{t[u/x] \rew a
      }{(\lambda x:U.t)~u \rew a}
    \qquad
    \ru{}{n~u \rew n~u}
  \)
  \end{center}

  \begin{center}
  \(
    \ru{t \rew (u;v) \qquad
        u \rew a
      }{t.1 \rew a}
    \qquad
    \ru{t \rew (u;v) \qquad
        v \rew a
      }{t.2 \rew a}
    \qquad
    \ru{t \rew n
      }{t.1 \rew n.1}
    \qquad
    \ru{t \rew n
      }{t.2 \rew n.2}
  \)
  \end{center}

  \begin{center}
  \(
    \ru{p \rew \refl_t \qquad
        u \rew a \qquad
        v \rew a \qquad
        t \rew a \qquad
        b\ u \rew a'
      }{\J (A,C,b,u,v,p) \rew a'}
  \)
  \end{center}

  \begin{center}
  \(
    \ru{p \rew n
      }{\J (A,C,b,u,v,p) \rew \J (A,C,b,u,v,n)}
  \)
  \end{center}
  %
  We will write $\red t$ for $a$ when $t \rew a$.

  \paradot{Type-directed Equality}

  \begin{center}
  \(
    \ru{\Gamma \der t \iff t' : \red T
      }{\Gamma \der t \hiff t' : T}
    \qquad
    \ru{\Gamma, x:U \der t~x \hiff t'~x : V
      }{\Gamma \der t \iff t' : \Pi x:U.V}
  \)
  \end{center}

  \begin{center}
  \(
    \ru{\Gamma \der t.1 \hiff t'.1 : U \qquad
        \Gamma \der t.2 \hiff t'.2 : V[t.1/x]
      }{\Gamma \der t \iff t' : \Sigma x:U.V}
  \)
  \end{center}

  \meta{We need observations for equality to define it, don't we?
  Otherwise we could just state equality for $\refl$.}

  \begin{center}
  \(
    \ru{\Gamma \der t \hiff t' : T
      }{\Gamma \der t \iff t' : \RRe{T}{U}{h}}
  \)
  \end{center}

  \paradot{Structural Equality}

  \begin{center}
  \(
    \ru{\Gamma \der n \hsteq n' : T
      }{\Gamma \der n \steq n' :\ \red T}
    \qquad
    \ru{(x:T) \in \Gamma
      }{\Gamma \der x \hsteq x : T}
  \)
  \end{center}

  \begin{center}
  \(
    \ru{\Gamma \der n \steq n' : \Pi x:U.V \qquad
        \Gamma \der u \hiff u' : U
      }{\Gamma \der n~u \hsteq n'~u' : V[u/x]}
  \)
  \end{center}

  \begin{center}
  \(
    \ru{\Gamma \der n \steq n' : \Sigma x:U.V
      }{\Gamma \der n.1 \hsteq n'.1 : U}
    \qquad
    \ru{\Gamma \der n \steq n' : \Sigma x:U.V
      }{\Gamma \der n.2 \hsteq n'.2 : V[n.1/x]}
  \)
  \end{center}

  \paradot{Type Equality}

  \begin{center}
  \(
    \ru{\Gamma \der \red T \iff \red T'
      }{\Gamma \der T \hiff T'}
    \qquad
    \ru{\Gamma \der U \hiff U' \qquad
        \Gamma, x:U \der V \hiff V'
      }{\Gamma \der \Pi x:U.V \iff \Pi x:U'.V'}
    \qquad
    \ru{\Gamma \der N \hsteq N' : \_
      }{\Gamma \der N \iff N'}
  \)
  \end{center}

  \begin{center}
  \(
    \ru{\Gamma \der U \hiff U' \qquad
        \Gamma, x:U \der V \hiff V'
      }{\Gamma \der \Sigma x:U.V \iff \Sigma x:U'.V'}
    \qquad
    \ru{\Gamma \der U \hiff U' \qquad
        \Gamma \der T \hiff T'
      }{\Gamma \der \RRe{U}{T}{h} \iff \RRe{U'}{T'}{h'}}
  \)
  \end{center}

  \begin{center}
  \(
    \ru{\Gamma \der T \hiff T' \qquad
        \Gamma \der t \hiff t' : T \qquad
        \Gamma \der u \hiff u' : T
      }{\Gamma \der t =_T u \iff t' =_T u'}
  \)
  \end{center}

  \section{A Logical Relation for Soundness}

  \paradot{An Induction Measure}

  In order to define a logical relation, we define semantic universe hierarchy.
  By recursion on $i \in \mathbb{N}$, we define
  $\U_i \in \Whnf \times \mathcal{P}(\Whnf)$ as follows.

  \begin{center}
  \(
    \ru{}{(N, \Wne) \in \U_i}
    \qquad
    \rux{}{(\Type_i, \mid \U_i \mid) \in \U_j}{(\Type_i,\Type_j) \in \Ax}
  \)
  \end{center}

  \begin{center}
  \(
    \rux{(U, \mathcal{A}) \in \widehat{U_i} \qquad
         \forall u \in \widehat{\mathcal{A}}.\ (T[u/x],\mathcal{F}(u)) \in
         \widehat{\U_j}
       }{(\Pi x:U.T, \Pi \mathcal{A} \mathcal{F}) \in \U_k
       }{(\Type_i, \Type_j, \Type_k) \in \R}
  \)
  \end{center}

  \begin{center}
  \(
    \rux{(U, \mathcal{A}) \in \widehat{\U_i} \qquad
        \forall u \in \widehat{\mathcal{A}},\ (V[u/x], \mathcal{F}(u)) \in
        \widehat{\U_j}
       }{(\Sigma x:U.V, \Sigma \mathcal{A} \mathcal{F}) \in \U_k
       }{(\Type_i, \Type_j, \Type_k) \in \R}
  \)
  \end{center}

  \meta{Here we also need to find a way to observe equalities to interperet
  the equaity type...}

  \begin{center}
  \(
    \ru{(U, \mathcal{A}) \in \widehat{\U_i}
      }{(\RRe{T}{U}{h}, \RRcal{\mathcal{A}}{h}) \in \U_i}
  \)
  \end{center}

  \noindent %
  Here, $\mathcal{A}$ denotes sets of expressions, $\mathcal{F}$ functions from
  expressions to set of expressions while
  $\widehat{\U_i} = \{ (T,\mathcal{A}) \mid (\red T, \mathcal{A}) \in \U_i \}$
  and $\mid \U_i \mid = \{ A \mid (A, \mathcal{A}) \in \U_i \text{ for some }
  \mathcal{A} \}$.
  $\widehat{\mathcal{A}} = \{ t \mid \red t \in \mathcal{A} \}$ is the closure
  of $\mathcal{A}$ by weak head expansion.
  The dependent function space is defined as
  $\Pi \mathcal{A} \mathcal{F} = \{ f \in \Whnf \mid \forall u \in
  \widehat{\mathcal{A}},\ f~u \in \widehat{\mathcal{F}(u)} \}$.
  The dependent sum space is defined as
  $\Sigma \mathcal{A} \mathcal{F} = \{ f \in \Whnf \mid f.1 \in
  \widehat{\mathcal{A}} \text{ and } f.2 \in \widehat{\mathcal{F}(f.1)} \}$.

  The resized space derived from an embedding is defined as
  $\RRcal{\mathcal{A}}{h} = \{ f \in \Whnf \mid h.1~f \in \widehat{\mathcal{A}}
  \}$.
  Note that intuitively, this corresponds to the interpretation of $T$
  which we can't mention yet for universe stratification reasons.
  With that in mind, we have $T \in \Type_i$ instead of $T \in \Type_j$ for
  a possibly larger $j$. This implies that, when considering the measure,
  we can make a recursive call on $T$ when it shouldn't be possible.
  \meta{I'm not sure of what I'm saying because this would work in a larger
  setting wouldn't it?}

  Throughout this paper, the induction measure $A \in \Type_i$ shall mean the
  minimum height of a derivation of $(A, \mathcal{A}) \in \U_i$ for some
  $\mathcal{A}$. Note that we also have that $A \in \Type_i$ is smaller
  than $\Type_i \in \Type_j$.
  We also write $A \in s$ for $A \in \Type_i$ for some $i$.

  \begin{proof}[The resized interpretation is identical]
    Say we have $(\RRe{A}{B}{h}, \RRcal{\mathcal{B}}{h}) \in \U_i$
    and $(A, \mathcal{A}) \in U_j$.
    Let's show $\RRcal{\mathcal{B}}{h} = \mathcal{A}$, meaning that they
    have the same elements.

    Let $f \in \mathcal{A}$ and show $f \in \RRcal{\mathcal{B}}{h}$, \ie
    show $h.1~f \in \mathcal{B}$.
    We should have that $h.1 \in \Pi \mathcal{A} \mathcal{F}$ where for any
    $u \in \widehat{\mathcal{A}}$, $\mathcal{F}(u) = \mathcal{B}$.
    So $h.1~f \in \mathcal{F}(f) = \mathcal{B}$.
    \meta{There is the problem of how to justify that $h.1$ indeed belongs to
    the interprertation of its type since we never required it to have one.}

    Now, conversely, assume $f \in \RRcal{\mathcal{B}}{h}$ and show
    $f \in \mathcal{A}$.%  We have $h.2.1 \in \Pi \mathcal{B} \mathcal{G}$
    % where $\mathcal{G}(v) = \mathcal{A}$ for any $v \in \mathcal{B}$.
    \meta{We need a way to give an interpretation to equality types if we want
    to go any further...}
  \end{proof}

  \paradot{A Kripke Logical Relation}

  Let $\Gamma \der t :=: t' : T$ stand for the conjunction of propositions
  \begin{itemize}
    \item $\Gamma \der t,t' : T$ and
    \item $\Gamma \der t = t' : T$.
  \end{itemize}
  %
  By induction on $A \in s$, we define two Kripke relations:
  \begin{itemize}
    \item $\Gamma \der A \Sr A' : s$
    \item $\Gamma \der a \Sr a' : A$
  \end{itemize}
  together with their respective closures $\hSr$.
  We define them in rule form for better readability meaning we have to see the
  conclusion to be defined as the conjunction of the premises.

  \begin{center}
  \(
    \ru{\Gamma \der N :=: N'
      }{\Gamma \der N \Sr N'}
    \qquad
    \ru{\Gamma \der n :=: n' : N
      }{\Gamma \der n \Sr n' : N}
    \qquad
    \rux{\der \Gamma
       }{\Gamma \der s \Sr s : s'
       }{(s,s')}
  \)
  \end{center}

  \begin{center}
  \(
    \trux{\Gamma \der U \hSr U' : s
        }{\forall \Delta \le \Gamma,\ \Delta \der u \hSr u' : U \gives
          \Delta \der V[u/x] \hSr V'[u'/x] : s'
        }{\Gamma \der \Pi x:U.V :=: \Pi x:U'.V' : s''
        }{\Gamma \der \Pi x:U.V \Sr \Pi x:U'.V' : s''
        }{(s,s',s'')}
  \)
  \end{center}

  \begin{center}
  \(
    \dru{\forall \Delta \le \Gamma,\ \Delta \der u \hSr u' : U \gives
         \Delta \der f~u \hSr f'~u' : V[u/x]
       }{\Gamma \der f :=: f' : \Pi x:U.V
       }{\Gamma \der f \Sr f' : \Pi x:U.V}
  \)
  \end{center}

  \begin{center}
  \(
    \trux{\Gamma \der U \hSr U' : s
        }{\forall \Delta \le \Gamma,\ \Delta \der u \hSr u' : U \gives
          \Delta \der V[u/x] \hSr V'[u'/x] : s'
        }{\Gamma \der \Sigma x:U.V :=: \Sigma x:U'.V' : s''
        }{\Gamma \der \Sigma x:U.V \Sr \Sigma x:U'.V' : s''
        }{(s,s',s'')}
  \)
  \end{center}

  \begin{center}
  \(
    \dru{\Gamma \der f.1 \hSr f'.1 : U \qquad
         \Gamma \der f.2 \hSr f'.2 : V[f.1/x]
       }{\Gamma \der f :=: f' : \Sigma x:U.V
       }{\Gamma \der f \Sr f' : \Sigma x:U.V}
  \)
  \end{center}

  \begin{center}
  \(
    \dru{\Gamma \der T \hSr T' : s \qquad
         \Gamma \der U \hSr U' : s'
       }{\Gamma \der h \hSr h : \emb{T}{U} \qquad
         \Gamma \der h' \hSr h' : \emb{T}{U}
       }{\Gamma \der \RRe{T}{U}{h} \Sr \RRe{T'}{U'}{h'} : s'}
  \)
  \end{center}

  \meta{If I'm not mistaken, this rule doesn't actually work, it would be nicer
  if we didn't need $\Gamma \der T \hSr T' : s$ to hold...}

  \begin{center}
  \(
    \dru{\Gamma \der f \hSr f' : T
       }{\Gamma \der f :=: f' : \RRe{T}{U}{h}
       }{\Gamma \der f \Sr f' : \RRe{T}{U}{h}}
  \)
  \end{center}

  \meta{This time it's the same, we can't talk about $T$ in the premises...}

  \begin{center}
  \(
    \qru{T \rew A \qquad
         \Gamma \der T = A
       }{t \rew a \qquad
         \Gamma \der t = a : A \qquad
         \Gamma \der t' = a' : A \qquad
         t' \rew a'
       }{\Gamma \der a \Sr a' : A
       }{\Gamma \der t :=: t' : T
       }{\Gamma \der t \hSr t' : T}
  \)
  \end{center}

\end{document}
