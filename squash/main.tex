\documentclass[a4paper,english]{lipics-utf8x}

\usepackage[T1]{fontenc} %

\usepackage{amsmath, amssymb, amsfonts, stmaryrd}
\usepackage{pifont}
\PrerenderUnicode{é} % For the author names in the heading

% Add some colors
\usepackage[usenames,dvipsnames,svgnames,table]{xcolor}
\usepackage{hyperref}
\hypersetup{
 linktocpage,
 colorlinks,
 citecolor=BlueViolet,
 filecolor=red,
 linkcolor=Blue,
 urlcolor=BrickRed
}

\usepackage{graphicx}
\usepackage{placeins}

% Meta comment
\newcommand\meta[1]{\noindent\textcolor{blue}{\emph{#1}}}

% Include the macro file
% evergreens
\newcommand{\der}{\,\vdash}
\newcommand{\Der}{\,\Vdash}

% semantic brackets
\def\lv{\mathopen{{[\kern-0.14em[}}}    % opening [[ value delimiter
\def\rv{\mathclose{{]\kern-0.14em]}}}   % closing ]] value delimiter
\newcommand{\den}[1]{\lv #1 \rv}
\newcommand{\Den}[3][]{\den{#2}^{#1}_{#3}}
\newcommand{\dent}[2]{\llparenthesis#1\rrparenthesis_{#2}}

% latin etc. abbrev
\newcommand{\abbrev}[1]{#1} % alternative: \emph{#1}
\newcommand{\cf}{\abbrev{cf.}\ }
\newcommand{\eg}{\abbrev{e.\,g.}}
\newcommand{\Eg}{\abbrev{E.\,g.}}
\newcommand{\ie}{\abbrev{i.\,e.}}
\newcommand{\Ie}{\abbrev{I.\,e.}}
\newcommand{\etal}{\abbrev{et.\,al.}}
\newcommand{\wwlog}{w.\,l.\,o.\,g.} % \wlog is ``write into log file''
\newcommand{\Wlog}{W.\,l.\,o.\,g.}
\newcommand{\wrt}{w.\,r.\,t.}

% Inference rules
\newcommand{\rulename}[1]{\ensuremath{\mbox{\sc#1}}}
\newcommand{\ru}[2]{\dfrac{\begin{array}[b]{@{}c@{}} #1 \end{array}}{#2}}
\newcommand{\rux}[3]{\ru{#1}{#2}~#3}
\newcommand{\nru}[3]{#1\ \ru{#2}{#3}}
\newcommand{\nrux}[4]{#1\ \ru{#2}{#3}\ #4}
\newcommand{\dstack}[2]{\begin{array}[b]{c}#1\\#2\end{array}}
\newcommand{\dru}[3]{\ru{\dstack{#1}{#2}}{#3}}
\newcommand{\tru}[4]{\dru{\dstack{#1}{#2}}{#3}{#4}}
\newcommand{\trux}[5]{\dru{\dstack{#1}{#2}}{#3}{#4}\ #5}
\newcommand{\qru}[5]{\tru{\dstack{#1}{#2}}{#3}{#4}{#5}}
\newcommand{\ndru}[4]{#1\ \ru{\dstack{#2}{#3}}{#4}}
\newcommand{\ndrux}[5]{#1\ \ru{\dstack{#2}{#3}}{#4}\ #5}

% Symbols and names
\newcommand\Type{\operatorname{Type}}
\newcommand\isnType[2]{\operatorname{is-}#1\operatorname{-Type}\ #2}
\newcommand\nType[1]{#1\operatorname{-Type}}
\newcommand\R{\operatorname{R}}
\newcommand\emb[2]{\operatorname{embedding}#1\ #2}
\newcommand\RRe[2]{\operatorname{RR_e}#1\ #2}
% \newcommand\type{\ \bm{\operatorname{type}}}
\DeclareMathOperator{\type}{\ \mathbf{type}}
\DeclareMathOperator{\ctr}{\mathbf{ctr}}
\DeclareMathOperator{\refl}{\mathbf{refl}}
\newcommand\rew{\searrow}
\newcommand\gettype{\operatorname{.type}}
\newcommand\getproof{\operatorname{.proof}}
\newcommand\Var{\operatorname{Var}}
\newcommand\Exp{\operatorname{Exp}}
\newcommand\Ctx{\operatorname{Ctx}}
\newcommand\Whnf{\operatorname{Whnf}}
\newcommand\Wne{\operatorname{Wne}}


% Title and so...
\title{Squash me to the Smallest Universe}
\author[1]{Théo Winterhalter}

\begin{document}

  \maketitle

  \begin{abstract}
    This time we propose a development of a squash lying in the smallest
    universe (or any universe for that matter).
  \end{abstract}

  \section{Syntax}

  \[
    \begin{array}{l@{~}l@{~}l@{~}r@{~}l@{\quad}l}
      \Var  & \ni & x,y,X,Y \\
      \Sort & \ni & s             & ::= & \Type_k \mbox{ }
                                                (k \in \mathbb{N}) \\
      \Exp  & \ni & t,u,T,U & ::= & s \mid \Pi x:U.T \mid \Sigma x:U.T \mid
                                    t =_T u \mid \squash{T} \\
                         &&& \mid & x \mid \lambda x:U.t \mid t~u
                               \mid (t;u) \mid t.1 \mid t.2 \mid \refl_t \mid
                               \J (T,U,t_{refl},u_1,u_2,t_{eq}) \\
                         &&& \mid & [t] \mid \elim(T,U,f,h,t) \\
      \Ctx  & \ni & \Gamma  & ::= & \cdot \mid \Gamma, x:T \\
    \end{array}
  \]

  \noindent %
  We write $A \to B$ as short for $\Pi \_:A.B$, the non-dependent product.
  We also write $=$ for $=_T$ when $T$ is understood.

  The \emph{novelty} here, with respect to MLTT, is the introduction of a squash
  $\squash{T}$ with an injection $[t]$ from $T$, all of its inhabitants being
  definitionally equal (which I believe is key). The only way to eliminate is
  in a (semantically) irrelevant context, \ie through something hProp.
  This will be translated by $\elim$.
  Basically $\elim(A,B,f,h,a)$ takes $f : A \to B$ with
  $h : \Pi x:B. \Pi y:B. x = y$ and $a : \squash{A}$ to return a $B$.

  \section{Typing Rules}

  \begin{center}
  \(
    \ru{}{\der \cdot}
    \qquad
    \ru{\Gamma \der T : s \qquad
        x \notin \Gamma
      }{\der \Gamma, x : T}
    \qquad
    \ru{\der \Gamma
      }{\Gamma \der \Type_i : \Type_{i+1}}
    \qquad
    \ru{\der \Gamma \qquad
        (x : T) \in \Gamma
      }{\Gamma \der x : T}
  \)
  \end{center}

  \begin{center}
  \(
    \ru{\Gamma \der t : \Pi x:A.B \qquad
        \Gamma \der t' : A
      }{\Gamma \der t\ t' : B[t'/x]}
    \qquad
    \ru{\Gamma \der A : s \qquad
        \Gamma, x:A \der B : s' \qquad
        (s,s',s'') \in \R
      }{\Gamma \der \Pi x:A.B : s''}
  \)
  \end{center}

  \begin{center}
  \(
    \ru{\Gamma \der \Pi x:A.B : s \qquad
        \Gamma, x:A \der t : B
      }{\Gamma \der \lambda x:A.t : \Pi x:A.B}
    \qquad
    \ru{\Gamma \der t : \Sigma x:A.B
      }{\Gamma \der t.1 : A}
    \qquad
    \ru{\Gamma \der t : \Sigma x:A.B
      }{\Gamma \der t.2 : B[t.1/x]}
  \)
  \end{center}

  \begin{center}
  \(
    \ru{\Gamma \der A : s \qquad
        \Gamma, x:A \der B : s' \qquad
        (s,s',s'') \in \R
      }{\Gamma \der \Sigma x:A.B : s''}
    \qquad
    \ru{\Gamma \der T : s \qquad
        \Gamma \der t,t' : T
      }{\Gamma \der t =_T t' : s}
  \)
  \end{center}

  \begin{center}
  \(
    \ru{\Gamma \der t : A \qquad
        \Gamma \der t' : B[t/x] \qquad
        \Gamma \der \Sigma x:A.B : s
      }{\Gamma \der (t;t') : \Sigma x:A.B}
    \qquad
    \ru{\Gamma \der t : T
      }{\Gamma \der \refl_t : t =_T t}
  \)
  \end{center}

  \begin{center}
  \(
    \tru{\Gamma \der A : s \qquad
         \Gamma \der C : \Pi x:A. \Pi y:A. (x =_A y) \to s'
       }{\Gamma \der b : \Pi x:A. C\ x\ x\ \refl_x
       }{\Gamma \der u, v : A \qquad
         \Gamma \der p : u =_A v
       }{\Gamma \der \J (A,C,b,u,v,p) : C\ u\ v\ p}
  \)
  \end{center}

  \begin{center}
  \(
    \ru{\Gamma \der t : A \qquad
        \Gamma \der B : s \qquad
        \Gamma \der A \le B
      }{\Gamma \der t : B}
  \)
  \end{center}

  \noindent %
  Now, we consider our rules for squash (acting as a resizing rule!).

  \begin{center}
  \(
    \ru{\Gamma \der A : \Type_i
      }{\Gamma \der \squash{A} : \Type_0}
    \qquad
    \ru{\Gamma \der t : A
      }{\Gamma \der [t] : \squash{A}}
  \)
  \end{center}

  \begin{center}
  \(
    \ru{\Gamma \der f : A \to B \qquad
        \Gamma \der h : \Pi_{x:A} \Pi_{y:A} x = y \qquad
        \Gamma \der a : \squash{A}
      }{\Gamma \der \elim(A,B,f,h,a) : B}
  \)
  \end{center}

  \section{Equality Rules}

  \paradot{Computation ($\beta$) and extensionality ($\eta$)}

  \begin{center}
  \(
    \ru{\Gamma, x:U \der t:V \qquad
        \Gamma \der u : U
      }{\Gamma \der (\lambda x:U.t)~u = t[u/x] : V[u/x]}
    \qquad
    \ru{\Gamma \der t : \Pi x:U.V
      }{\Gamma \der t = \lambda x:U.t~x : \Pi x:U.V}
  \)
  \end{center}

  \begin{center}
  \(
    \ru{\Gamma \der t : U \qquad
        \Gamma \der t' : V[t/x]
      }{\Gamma \der (t;t').1 = t : U}
    \qquad
    \ru{\Gamma \der t : U \qquad
        \Gamma \der t' : V[t/x]
      }{\Gamma \der (t;t').2 = t' : V[t/x]}
  \)
  \end{center}

  \begin{center}
  \(
    \ru{\Gamma \der t : \Sigma x:U.T
      }{\Gamma \der t = (t.1 ; t.2) : \Sigma x:U.T}
  \)
  \end{center}

  \begin{center}
  \(
    \dru{\Gamma \der A : s \qquad
         \Gamma \der C : \Pi x:A. \Pi y:A. (x =_A y) \to s'
       }{\Gamma \der b : \Pi x:A. C\ x\ x\ \refl_x \qquad
         \Gamma \der u : A
       }{\Gamma \der \J (A,C,b,u,u,\refl_u) = b\ u : C\ u\ u\ \refl_u}
  \)
  \end{center}

  \begin{center}
  \(
    \ru{\Gamma \der f : A \to B \qquad
        \Gamma \der h : \Pi_{x:A} \Pi_{y:A} x = y \qquad
        \Gamma \der t : A
      }{\Gamma \der \elim(A,B,f,h,[t]) = f\ t : B}
  \)
  \end{center}

  \paradot{Equivalence Rules}

  \begin{center}
  \(
    \ru{\Gamma \der t : T
      }{\Gamma \der t = t : T}
    \qquad
    \ru{\Gamma \der t' = t : T
      }{\Gamma \der t = t' : T}
    \qquad
    \ru{\Gamma \der t_1 = t_2 : T \qquad
        \Gamma \der t_2 = t_3 : T
      }{\Gamma \der t_1 = t_3 : T}
  \)
  \end{center}

  \paradot{Compatibility Rules}

  \begin{center}
  \(
    \rux{\Gamma \der U = U' : s \qquad
         \Gamma, x:U \der V = V' : s'
       }{\Gamma \der \Pi x:U.V = \Pi x:U'.V' : s''
       }{(s,s',s'')}
  \)
  \end{center}

  \begin{center}
  \(
    \ru{\Gamma \der U = U' : s \qquad
        \Gamma, x:U \der V : s' \qquad
        \Gamma, x:U \der t = t' : V
      }{\Gamma \der \lambda x:U.t = \lambda x:U'.t' : \Pi x:U.V}
  \)
  \end{center}

  \begin{center}
  \(
    \ru{\Gamma \der t = t' : \Pi x:U.V \qquad
        \Gamma \der u = u' : U
      }{\Gamma \der t~u = t'~u' : V[u/x]}
  \)
  \end{center}

  \begin{center}
  \(
    \rux{\Gamma \der U = U' : s \qquad
         \Gamma, x:U \der V = V' : s'
       }{\Gamma \der \Sigma x:U.V = \Sigma x:U'.V' : s''
       }{(s,s',s'')}
  \)
  \end{center}

  \begin{center}
  \(
    \ru{\Gamma \der t_1 = t'_1 : U \qquad
        \Gamma, x:U \der V : s \qquad
        \Gamma \der t_2 = t'_2 : V[t_1/x]
      }{\Gamma \der (t_1;t_2) = (t'_1;t'_2) : \Sigma x:U.V}
  \)
  \end{center}

  \begin{center}
  \(
    \ru{\Gamma \der t = t' : \Sigma x:U.V
      }{\Gamma \der t.1 = t'.1 : U}
    \qquad
    \ru{\Gamma \der t = t' : \Sigma x:U.V
      }{\Gamma \der t.2 = t'.2 : V[t.1/x]}
  \)
  \end{center}

  \begin{center}
  \(
    \ru{\Gamma \der T = T' \qquad
        \Gamma \der t = t' : T \qquad
        \Gamma \der u = u' : T
      }{\Gamma \der t =_T u = t' =_{T'} u'}
  \)
  \end{center}

  \begin{center}
  \(
    \ru{\Gamma \der t = t' : T \qquad
      }{\Gamma \der \refl_t = \refl_{t'} : t =_T t}
  \)
  \end{center}

  \begin{center}
  \(
    \tru{\Gamma \der A = A' : s \qquad
         \Gamma \der C = C' : \Pi x:A. \Pi y:A. (x =_A y) \to s'
       }{\Gamma \der b = b' : \Pi x:A. C\ x\ x\ \refl_x
       }{\Gamma \der u = u' : A \qquad
         \Gamma \der v = v' : A \qquad
         \Gamma \der p = p' : u =_A v
       }{\Gamma \der \J (A,C,b,u,v,p) = \J (A',C',b',u',v',p') : C\ u\ v\ p}
  \)
  \end{center}

  \begin{center}
  \(
    \ru{\Gamma \der A = A' : s
      }{\Gamma \der \squash{A} = \squash{A'} : \Type_0}
    \qquad
    \ru{\Gamma \der t, t' : \squash{A}
      }{\Gamma \der t = t' : \squash{A}}
  \)
  \end{center}

  \begin{center}
  \(
    \dru{\Gamma \der A = A' : s \qquad
         \Gamma \der B = B' : s'
       }{\Gamma \der f = f' : A \to B \qquad
         \Gamma \der h, h' : \Pi_{x:A} \Pi_{y:A} x = y \qquad
         \Gamma \der a, a' : \squash{A}
       }{\Gamma \der \elim(A,B,f,h,a) = \elim(A,B,f,h,a) : B}
  \)
  \end{center}

  \paradot{Conversion Rule}

  \begin{center}
  \(
    \ru{\Gamma \der t = t' : T \quad
        \Gamma \der T \le T'
      }{\Gamma \der t = t' : T'}
  \)
  \end{center}

  \section{Cumulativity}

  \begin{center}
  \(
    \rux{}{\Gamma \der \Type_i \le \Type_j}{i \le j}
    \qquad
    \ru{\Gamma \der U = U' : s \qquad
        \Gamma, x:U \der T \le T'
      }{\Gamma \der \Pi x:U.T \le \Pi x:U'.T'}
  \)
  \end{center}

  \begin{center}
  \(
    \ru{\Gamma \der T = T' : s
      }{\Gamma \der T \le T'}
    \qquad
    \ru{\Gamma \der T_1 \le T_2 \qquad
        \Gamma \der T_2 \le T_3
      }{\Gamma \der T_1 \le T_3}
  \)
  \end{center}

\end{document}
