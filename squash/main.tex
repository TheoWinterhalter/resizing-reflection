\documentclass[a4paper,english]{lipics-utf8x}

\usepackage[T1]{fontenc} %

\usepackage{amsmath, amssymb, amsfonts, stmaryrd}
\usepackage{pifont}
\PrerenderUnicode{é} % For the author names in the heading

% Add some colors
\usepackage[usenames,dvipsnames,svgnames,table]{xcolor}
\usepackage{hyperref}
\hypersetup{
 linktocpage,
 colorlinks,
 citecolor=BlueViolet,
 filecolor=red,
 linkcolor=Blue,
 urlcolor=BrickRed
}

\usepackage{graphicx}
\usepackage{placeins}

% Meta comment
\newcommand\meta[1]{\noindent\textcolor{blue}{\emph{#1}}}

% Include the macro file
% evergreens
\newcommand{\der}{\,\vdash}
\newcommand{\Der}{\,\Vdash}

% semantic brackets
\def\lv{\mathopen{{[\kern-0.14em[}}}    % opening [[ value delimiter
\def\rv{\mathclose{{]\kern-0.14em]}}}   % closing ]] value delimiter
\newcommand{\den}[1]{\lv #1 \rv}
\newcommand{\Den}[3][]{\den{#2}^{#1}_{#3}}
\newcommand{\dent}[2]{\llparenthesis#1\rrparenthesis_{#2}}

% latin etc. abbrev
\newcommand{\abbrev}[1]{#1} % alternative: \emph{#1}
\newcommand{\cf}{\abbrev{cf.}\ }
\newcommand{\eg}{\abbrev{e.\,g.}}
\newcommand{\Eg}{\abbrev{E.\,g.}}
\newcommand{\ie}{\abbrev{i.\,e.}}
\newcommand{\Ie}{\abbrev{I.\,e.}}
\newcommand{\etal}{\abbrev{et.\,al.}}
\newcommand{\wwlog}{w.\,l.\,o.\,g.} % \wlog is ``write into log file''
\newcommand{\Wlog}{W.\,l.\,o.\,g.}
\newcommand{\wrt}{w.\,r.\,t.}

% Inference rules
\newcommand{\rulename}[1]{\ensuremath{\mbox{\sc#1}}}
\newcommand{\ru}[2]{\dfrac{\begin{array}[b]{@{}c@{}} #1 \end{array}}{#2}}
\newcommand{\rux}[3]{\ru{#1}{#2}~#3}
\newcommand{\nru}[3]{#1\ \ru{#2}{#3}}
\newcommand{\nrux}[4]{#1\ \ru{#2}{#3}\ #4}
\newcommand{\dstack}[2]{\begin{array}[b]{c}#1\\#2\end{array}}
\newcommand{\dru}[3]{\ru{\dstack{#1}{#2}}{#3}}
\newcommand{\tru}[4]{\dru{\dstack{#1}{#2}}{#3}{#4}}
\newcommand{\trux}[5]{\dru{\dstack{#1}{#2}}{#3}{#4}\ #5}
\newcommand{\qru}[5]{\tru{\dstack{#1}{#2}}{#3}{#4}{#5}}
\newcommand{\ndru}[4]{#1\ \ru{\dstack{#2}{#3}}{#4}}
\newcommand{\ndrux}[5]{#1\ \ru{\dstack{#2}{#3}}{#4}\ #5}

% Symbols and names
\newcommand\Type{\operatorname{Type}}
\newcommand\isnType[2]{\operatorname{is-}#1\operatorname{-Type}\ #2}
\newcommand\nType[1]{#1\operatorname{-Type}}
\newcommand\R{\operatorname{R}}
\newcommand\emb[2]{\operatorname{embedding}#1\ #2}
\newcommand\RRe[2]{\operatorname{RR_e}#1\ #2}
% \newcommand\type{\ \bm{\operatorname{type}}}
\DeclareMathOperator{\type}{\ \mathbf{type}}
\DeclareMathOperator{\ctr}{\mathbf{ctr}}
\DeclareMathOperator{\refl}{\mathbf{refl}}
\newcommand\rew{\searrow}
\newcommand\gettype{\operatorname{.type}}
\newcommand\getproof{\operatorname{.proof}}
\newcommand\Var{\operatorname{Var}}
\newcommand\Exp{\operatorname{Exp}}
\newcommand\Ctx{\operatorname{Ctx}}
\newcommand\Whnf{\operatorname{Whnf}}
\newcommand\Wne{\operatorname{Wne}}


% Title and so...
\title{The squash is small}
\author[1]{Théo Winterhalter}

\begin{document}

  \maketitle

  \begin{abstract}
    This time we propose a development of a squash lying in the smallest
    universe (or any universe for that matter).
  \end{abstract}

  \section{Syntax}

  \[
    \begin{array}{l@{~}l@{~}l@{~}r@{~}l@{\quad}l}
      \Var  & \ni & x,y,X,Y \\
      \Sort & \ni & s             & ::= & \Type_k \mbox{ }
                                                (k \in \mathbb{N}) \\
      \Exp  & \ni & t,u,T,U & ::= & s \mid \Pi x:U.T \mid
                                    \Id T\ t\ u \mid \squash{T} \\
                         &&& \mid & x \mid \lambda x:U.t \mid t~u
                               \mid \refl_t \mid
                               \J (T,U,t_{refl},u_1,u_2,t_{eq}) \\
                         &&& \mid & [t] \mid \elim(T,U,f,h,t)
                             \mid \seq(T,t,u) \\
      \Ctx  & \ni & \Gamma  & ::= & \cdot \mid \Gamma, x:T \\
    \end{array}
  \]

  \noindent %
  We write $A \to B$ as short for $\Pi \_:A.B$, the non-dependent product.

  The \emph{novelty} here, with respect to MLTT, is the introduction of a squash
  $\squash{T}$ with an injection $[t]$ from $T$, all of its inhabitants being
  definitionally equal. The only way to eliminate is in a (semantically)
  irrelevant context, \ie through something hProp.
  This will be translated by $\elim$.
  Basically $\elim(A,B,f,h,a)$ takes $f : A \to B$ with
  $h : \Pi x:B. \Pi y:B. \Id B\ x\ y$ and $a : \squash{A}$ to return a $B$
  as long as $B$ is not a sort (and thus, not a variable).

  \section{Typing Rules}

  \begin{center}
  \(
    \ru{}{\der \cdot}
    \qquad
    \ru{\Gamma \der T : s \qquad
        x \notin \Gamma
      }{\der \Gamma, x : T}
    \qquad
    \ru{\der \Gamma
      }{\Gamma \der \Type_i : \Type_{i+1}}
    \qquad
    \ru{\der \Gamma \qquad
        (x : T) \in \Gamma
      }{\Gamma \der x : T}
  \)
  \end{center}

  \begin{center}
  \(
    \ru{\Gamma \der t : \Pi x:A.B \qquad
        \Gamma \der t' : A
      }{\Gamma \der t\ t' : B[t'/x]}
    \qquad
    \ru{\Gamma \der A : s \qquad
        \Gamma, x:A \der B : s' \qquad
        (s,s',s'') \in \R
      }{\Gamma \der \Pi x:A.B : s''}
  \)
  \end{center}

  \begin{center}
  \(
    \ru{\Gamma \der \Pi x:A.B : s \qquad
        \Gamma, x:A \der t : B
      }{\Gamma \der \lambda x:A.t : \Pi x:A.B}
    \qquad
    \ru{\Gamma \der T : s \qquad
        \Gamma \der t,t' : T
      }{\Gamma \der \Id T\ t\ t' : s}
  \)
  \end{center}

  \begin{center}
  \(
    \ru{\Gamma \der t : T
      }{\Gamma \der \refl_t : \Id T\ t\ t}
    \qquad
    \tru{\Gamma \der A : s \qquad
         \Gamma \der C : \Pi x:A. \Pi y:A. (\Id A\ x\ y) \to s'
       }{\Gamma \der b : \Pi x:A. C\ x\ x\ \refl_x
       }{\Gamma \der u, v : A \qquad
         \Gamma \der p : \Id A\ u\ v
       }{\Gamma \der \J (A,C,b,u,v,p) : C\ u\ v\ p}
  \)
  \end{center}

  \begin{center}
  \(
    \ru{\Gamma \der t : A \qquad
        \Gamma \der B : s \qquad
        \Gamma \der A \le B
      }{\Gamma \der t : B}
  \)
  \end{center}

  \noindent %
  Now, we consider our rules for squash (acting as a resizing rule!).
  Elimination requires that $B$ is not a sort in a positive way:
  $B \nequiv s$ shall stand for
  $B ::\equiv \Pi \_ \mid \Id \_ \mid \squash{\_}$.

  \begin{center}
  \(
    \ru{\Gamma \der A : \Type_i
      }{\Gamma \der \squash{A} : \Type_0}
    \qquad
    \ru{\Gamma \der t : A
      }{\Gamma \der [t] : \squash{A}}
    \qquad
    \ru{\Gamma \der T : s \qquad
        \Gamma \der u,v : \squash{T}
      }{\Gamma \der \seq(T,u,v) : \Id \squash{T}\ u\ v}
  \)
  \end{center}

  \begin{center}
  \(
    \ru{\Gamma \der f : A \to B \qquad
        \Gamma \der h : \Pi_{x:A} \Pi_{y:A} \Id A\ x\ y \qquad
        \Gamma \der a : \squash{A} \qquad
        B \nequiv s
      }{\Gamma \der \elim(A,B,f,h,a) : B}
  \)
  \end{center}

  \section{Equality Rules}

  \paradot{Computation ($\beta$) and extensionality ($\eta$)}

  \begin{center}
  \(
    \ru{\Gamma, x:U \der t:V \qquad
        \Gamma \der u : U
      }{\Gamma \der (\lambda x:U.t)~u = t[u/x] : V[u/x]}
    \qquad
    \ru{\Gamma \der t : \Pi x:U.V
      }{\Gamma \der t = \lambda x:U.t~x : \Pi x:U.V}
  \)
  \end{center}

  \begin{center}
  \(
    \ru{\Gamma \der t : U \qquad
        \Gamma \der t' : V[t/x]
      }{\Gamma \der (t;t').1 = t : U}
    \qquad
    \ru{\Gamma \der t : U \qquad
        \Gamma \der t' : V[t/x]
      }{\Gamma \der (t;t').2 = t' : V[t/x]}
  \)
  \end{center}

  \begin{center}
  \(
    \ru{\Gamma \der t : \Sigma x:U.T
      }{\Gamma \der t = (t.1 ; t.2) : \Sigma x:U.T}
  \)
  \end{center}

  \begin{center}
  \(
    \dru{\Gamma \der A : s \qquad
         \Gamma \der C : \Pi x:A. \Pi y:A. (\Id A\ x\ y) \to s'
       }{\Gamma \der b : \Pi x:A. C\ x\ x\ \refl_x \qquad
         \Gamma \der u : A
       }{\Gamma \der \J (A,C,b,u,u,\refl_u) = b\ u : C\ u\ u\ \refl_u}
  \)
  \end{center}

  \begin{center}
  \(
    \ru{\Gamma \der f : A \to B \qquad
        \Gamma \der h : \Pi_{x:A} \Pi_{y:A} \Id A\ x\ y \qquad
        \Gamma \der t : A \qquad
        B \nequiv s
      }{\Gamma \der \elim(A,B,f,h,[t]) = f\ t : B}
  \)
  \end{center}

  \paradot{Equivalence Rules}

  \begin{center}
  \(
    \ru{\Gamma \der t : T
      }{\Gamma \der t = t : T}
    \qquad
    \ru{\Gamma \der t' = t : T
      }{\Gamma \der t = t' : T}
    \qquad
    \ru{\Gamma \der t_1 = t_2 : T \qquad
        \Gamma \der t_2 = t_3 : T
      }{\Gamma \der t_1 = t_3 : T}
  \)
  \end{center}

  \paradot{Compatibility Rules}

  \begin{center}
  \(
    \rux{\Gamma \der U = U' : s \qquad
         \Gamma, x:U \der V = V' : s'
       }{\Gamma \der \Pi x:U.V = \Pi x:U'.V' : s''
       }{(s,s',s'')}
  \)
  \end{center}

  \begin{center}
  \(
    \ru{\Gamma \der U = U' : s \qquad
        \Gamma, x:U \der V : s' \qquad
        \Gamma, x:U \der t = t' : V
      }{\Gamma \der \lambda x:U.t = \lambda x:U'.t' : \Pi x:U.V}
  \)
  \end{center}

  \begin{center}
  \(
    \ru{\Gamma \der t = t' : \Pi x:U.V \qquad
        \Gamma \der u = u' : U
      }{\Gamma \der t~u = t'~u' : V[u/x]}
  \)
  \end{center}

  \begin{center}
  \(
    \rux{\Gamma \der U = U' : s \qquad
         \Gamma, x:U \der V = V' : s'
       }{\Gamma \der \Sigma x:U.V = \Sigma x:U'.V' : s''
       }{(s,s',s'')}
  \)
  \end{center}

  \begin{center}
  \(
    \ru{\Gamma \der t_1 = t'_1 : U \qquad
        \Gamma, x:U \der V : s \qquad
        \Gamma \der t_2 = t'_2 : V[t_1/x]
      }{\Gamma \der (t_1;t_2) = (t'_1;t'_2) : \Sigma x:U.V}
  \)
  \end{center}

  \begin{center}
  \(
    \ru{\Gamma \der t = t' : \Sigma x:U.V
      }{\Gamma \der t.1 = t'.1 : U}
    \qquad
    \ru{\Gamma \der t = t' : \Sigma x:U.V
      }{\Gamma \der t.2 = t'.2 : V[t.1/x]}
  \)
  \end{center}

  \begin{center}
  \(
    \ru{\Gamma \der T = T' : s \qquad
        \Gamma \der t = t' : T \qquad
        \Gamma \der u = u' : T
      }{\Gamma \der \Id T\ t\ u = \Id T'\ t'\ u' : s}
  \)
  \end{center}

  \begin{center}
  \(
    \ru{\Gamma \der t = t' : T \qquad
      }{\Gamma \der \refl_t = \refl_{t'} : \Id T\ t\ t}
  \)
  \end{center}

  \begin{center}
  \(
    \tru{\Gamma \der A = A' : s \qquad
         \Gamma \der C = C' : \Pi x:A. \Pi y:A. (\Id A\ x\ y) \to s'
       }{\Gamma \der b = b' : \Pi x:A. C\ x\ x\ \refl_x
       }{\Gamma \der u = u' : A \qquad
         \Gamma \der v = v' : A \qquad
         \Gamma \der p = p' : \Id A\ u\ v
       }{\Gamma \der \J (A,C,b,u,v,p) = \J (A',C',b',u',v',p') : C\ u\ v\ p}
  \)
  \end{center}

  \begin{center}
  \(
    \ru{\Gamma \der A = A' : s
      }{\Gamma \der \squash{A} = \squash{A'} : \Type_0}
    \qquad
    \ru{\Gamma \der T = T' : s \qquad
        \Gamma \der t = t' : \squash{T} \qquad
        \Gamma \der u = u' : \squash{T}
      }{\Gamma \der \seq(T,t,u) = \seq(T',t',u') : \Id \squash{T}\ t\ u}
  \)
  \end{center}

  \begin{center}
  \(
    \dru{\Gamma \der A = A' : s \qquad
         \Gamma \der B = B' : s' \qquad
         B \nequiv s
       }{\Gamma \der f = f' : A \to B \qquad
         \Gamma \der h, h' : \Pi_{x:A} \Pi_{y:A} \Id A\ x\ y \qquad
         \Gamma \der a, a' : \squash{A}
       }{\Gamma \der \elim(A,B,f,h,a) = \elim(A,B,f,h,a) : B}
  \)
  \end{center}

  \paradot{Conversion Rule}

  \begin{center}
  \(
    \ru{\Gamma \der t = t' : T \quad
        \Gamma \der T \le T'
      }{\Gamma \der t = t' : T'}
  \)
  \end{center}

  \section{Cumulativity}

  \begin{center}
  \(
    \rux{}{\Gamma \der \Type_i \le \Type_j}{i \le j}
    \qquad
    \ru{\Gamma \der U = U' : s \qquad
        \Gamma, x:U \der T \le T'
      }{\Gamma \der \Pi x:U.T \le \Pi x:U'.T'}
  \)
  \end{center}

  \begin{center}
  \(
    \ru{\Gamma \der T = T' : s
      }{\Gamma \der T \le T'}
    \qquad
    \ru{\Gamma \der T_1 \le T_2 \qquad
        \Gamma \der T_2 \le T_3
      }{\Gamma \der T_1 \le T_3}
  \)
  \end{center}

  \section{Consistency}

  While, initially, the Kripke logical relation we're going to define was
  created to prove soundness of algorithmic equality, we will mainly focus
  on consistency. We might handle decidability later.

  \paradot{Weak head normalization}

  Weak head normal forms (whnfs) are given by the following grammar:

  \begin{align*}
    \Whnf &\ni a,f,A,B,F &::=~& \Pi x:U.T \mid \Id T\ t\ u \mid \squash{T} \\
        &&\mid~& n \mid \lambda x:U.t \mid \refl_t \mid [t] \mid \seq(T,t,u) \\
    \Wne  &\ni n,N &::=~& x \mid n~u \mid \elim(T,U,t,u,n) \mid \J (T,U,b,u,v,n)
  \end{align*}
  %
  We present weak-head reduction as follows:

  \begin{center}
  \(
    \ru{t \rew f \qquad
        f~u \rew a
      }{t~u \rew a}
    \qquad
    \ru{}{a \rew a}
    \qquad
    \ru{t[u/x] \rew a
      }{(\lambda x:U.t)~u \rew a}
    \qquad
    \ru{}{n~u \rew n~u}
  \)
  \end{center}

  \begin{center}
  \(
    \ru{p \rew \refl_t \qquad
        u \rew a \qquad
        v \rew a \qquad
        t \rew a \qquad
        b\ u \rew a'
      }{\J (T,U,b,u,v,p) \rew a'}
  \)
  \end{center}

  \begin{center}
  \(
    \ru{p \rew n
      }{\J (T,U,b,u,v,p) \rew \J (T,U,b,u,v,n)}
  \)
  \end{center}

  \begin{center}
  \(
    \ru{v \rew [v'] \qquad
        t\ v' \rew a
      }{\elim(T,U,t,u,v) \rew a}
    \qquad
    \ru{v \rew n
      }{\elim(T,U,t,u,v) \rew \elim(T,U,t,u,n)}
  \)
  \end{center}
  %
  We will write $\red t$ for $a$ when $t \rew a$.

  \paradot{An Induction Measure}

  In order to define the logical relation, we define semantic universe
  hierarchy.
  By recursion on $i \in \mathbb{N}$, we define
  $\U_i \in \Whnf \times \mathcal{P}(\Whnf)$ as follows.

  \begin{center}
  \(
    \ru{}{(N, \Wne) \in \U_i}
    \qquad
    \rux{}{(\Type_i, \mid \U_i \mid) \in \U_j}{(\Type_i,\Type_j)}
  \)
  \end{center}

  \begin{center}
  \(
    \rux{(U, \mathcal{A}) \in \widehat{U_i} \qquad
         \forall u \in \widehat{\mathcal{A}}.\ (T[u/x],\mathcal{F}(u)) \in
         \widehat{\U_j}
       }{(\Pi x:U.T, \Pi \mathcal{A} \mathcal{F}) \in \U_k
       }{(\Type_i, \Type_j, \Type_k)}
  \)
  \end{center}

  \begin{center}
  \(
    \ru{(T,\mathcal{A}) \in \U_i
      }{(\Id T\ u\ v, \oeq\ \mathcal{A}\ u\ v) \in \U_i}
    \qquad
    \ru{}{(\squash{A}, \Squash) \in \U_i}
  \)
  \end{center}

  \noindent %
  Here, $\mathcal{A}$ denotes sets of expressions, $\mathcal{F}$ functions from
  expressions to set of expressions while
  $\widehat{\U_i} = \{ (T,\mathcal{A}) \mid (\red T, \mathcal{A}) \in \U_i \}$
  and $\mid \U_i \mid = \{ A \mid (A, \mathcal{A}) \in \U_i \text{ for some }
  \mathcal{A} \}$.
  $\widehat{\mathcal{A}} = \{ t \mid \red t \in \mathcal{A} \}$ is the closure
  of $\mathcal{A}$ by weak head expansion.
  The dependent function space is defined as
  $\Pi \mathcal{A} \mathcal{F} = \{ f \in \Whnf \mid \forall u \in
  \widehat{\mathcal{A}},\ f~u \in \widehat{\mathcal{F}(u)} \}$.
  The equality space is defined as
  $\oeq\ \mathcal{A}\ u\ v = \Wne \bigcup \{ \refl_t \mid t \in
  \widehat{\mathcal{A}} \}$.
  Finally, the squash space is just
  $\Squash = \Wne \bigcup \{ [t] \mid \red t \in \Whnf \}$.

  \paradot{Transports}

  In order to define the logical relation, we first need to define some
  transports that will make sure everything is well-typed. We indeed need to
  since our logical relation doesn't only capture conversion but all provable
  equalities.

  \[\transport : \Pi A\ A' : \Type.(\Id \Type\ A\ A') \to A \to A'\]
  \[\transport A\ A'\ p\ x := \J(\Type, (\lambda T\ T'\ eq. T \to T'),
  (\lambda T\ x. x), A, A', p)\ x\]
  %
  We will write $p_*$ for $\transport A\ A'\ p$.

  \[\ap : \Pi A\ B : \Type. \Pi (f : A \to B). \Pi (u\ v : A).
    (\Id u\ v) \to (\Id (f\ u)\ (f\ v))\]
  \[\ap A\ B\ f\ u\ v\ p := \J(A, (\lambda x\ y\ p. \Id (f\ x)\ (f\ y)),
  (\lambda x. \refl_{f\ x}), u, v, p)\]
  %
  We will write $\ap f : \Id x\ y \to \Id (f\ x)\ (f\ y)$ with the other
  arguments understood.

  \[\sym : \Pi A : \Type. \Pi x\ y : A. (\Id x\ y) \to (\Id y\ x)\]
  \[\sym A\ x\ y\ p := \J(A, (\lambda u\ v\ e. \Id v\ u), (\lambda u. \refl_u),
  x,y,p)\]
  %
  We will write $p\inv$ for $\sym A\ x\ y\ p$.

  \[\trans : \Pi A : \Type. \Pi x\ y\ z : A.
  (\Id x\ y) \to (\Id y\ z) \to (\Id x\ z)\]
  \[
    \trans A\ x\ y\ z\ p := \J(A, (\lambda u\ v\ e. \Id v\ z \to \Id u\ z),
    (\lambda u\ e. e), x, y, p)
  \]
  %
  We will write $p \cons q$ for $\trans A\ x\ y\ z\ p\ q$.

  \[
    \transdom : \Pi (A\ A' : \Type)\ (p : \Id A\ A')\ (B' : A' \to \Type).
                (A \to \Type)
  \]
  \[
    \transdom p\ B'\ a := B'\ (p_* a)
  \]
  As in the definition, we will omit $A$ and $A'$.

  \[
  \begin{array}{l@{\quad}l}
    \transfun :~& \Pi (A\ A' : \Type)\ (B : A \to \Type)\ (B' : A' \to \Type) \\
                & (p : \Id A\ A').\ \Id B\ (\transdom p\ B') \to \\
                & \Id (\Pi (x:A).B) (\Pi (x:A').B') \\
    \transfun :=& ...
  \end{array}
  \]
  It is done in agda, let's not write it.
  We will write $\pi\ p\ q$ for $\transfun A\ A'\ B\ B'\ p\ q$.
  We will admit we have a similar $\id\ p\ q\ r$ for identity types and
  $\squash{p}$ for squashes.

  \[
  \begin{array}{l@{\quad}l}
    \app :~& \Pi (A : \Type)\ (B : A \to \Type)\ (f\ f' : \Pi x:A.B) \\
           & (u\ u' : A)\ (p : \Id f\ f')\ (q : \Id u\ u') \to \\
           & \Id (f\ u)\ (((\ap (\lambda x.B)\ (q\inv))_* (f'\ u')))
  \end{array}
  \]

  \paradot{A Kripke Logical Relation}

  By induction on $A \in s$, we define two Kripke relations:
  \begin{center}
  \(
    \Gamma \der \hideq{p} A \Er A' : s
    \qquad
    \Gamma \der \hideq{p} a \Er a' : A
  \)
  \end{center}
  together with their respective closures $\hEr$.
  We define them in rule form for better readability meaning we have to see the
  conclusion to be defined as the conjunction of the premises.

  \begin{mathc}
    \ru{\Gamma \der e : \Id A\ a\ a' \qquad
        A \nequiv s
      }{\Gamma \der \hideq{e} a \Er a' : A}
    \qquad
    \ru{\Gamma \der \hideq{p} \red t \Er \red t' : \red T
      }{\Gamma \der \hideq{p} t \hEr t' : T}
  \end{mathc}

  \begin{center}
  \(
    \ru{\Gamma \der e : \Id s\ N\ N'
      }{\Gamma \der \hideq{e} N \Er N' : s}
    \qquad
    \ru{\Gamma \der e : \Id s''\ s\ s'
      }{\Gamma \der \hideq{e} s \Er s' : s''}
    \qquad
    \ru{\Gamma \der e : \Id s\ T\ T'
      }{\Gamma \der \hideq{\squash{e}} \squash{T} \Er \squash{T'} : s'}
  \)
  \end{center}

  \begin{center}
  \(
    \rux{\Gamma \der \hideq{p} U \hEr U' : s \qquad
          \Gamma, x:U \der \hideq{q} V \hEr V'[p_*\ x / x] : s'
        }{\Gamma \der \hideq{\pi\ p\ q} \Pi x:U.V \Er \Pi x:U'.V' : s''
        }{(s,s',s'')}
  \)
  \end{center}

  \begin{mathc}
    \ru{\Gamma \der \hideq{p} T \hEr T' : s \qquad
        \Gamma \der \hideq{q} u \hEr (p\inv)_*\ u' : T \qquad
        \Gamma \der \hideq{r} v \hEr (p\inv)_*\ v' : T
      }{\Gamma \der \hideq{\id\ p\ q\ r} \Id T\ u\ v \Er \Id T'\ u'\ v' : s}
  \end{mathc}

  \begin{lemma}[Weakening]
    \leavevmode
    \begin{itemize}
      \item If $\Gamma \der \hideq{p} a \Er a' : A$ and $\Delta \le \Gamma$ then
      there exists a derivation of $\Delta \der \hideq{p} a \Er a' : A$ with the
      same height.
      \item Analogously for $\Gamma \der \hideq{p} t \hEr t' : T$.
    \end{itemize}
  \end{lemma}
  %
  \begin{proof}
    By induction on $A \in s$ and $T \in s$.
  \end{proof}

  \begin{lemma}[Type Conversion]
    \label{lem:s-conv}
    \leavevmode
    \begin{itemize}
      \item If $\Gamma \der \hideq{p} A \Er A' : s$ and
      $\Gamma \der \hideq{q} a \Er a' : A$ then
      $\Gamma \der \hideq{\ap p_*\ q} p_*\ a \Er p_*\ a' : A'$.
      \item If $\Gamma \der \hideq{p} T \hEr T' : s$ and
      $\Gamma \der \hideq{q} t \hEr t : T$ then
      $\Gamma \der \hideq{\ap p_*\ q} p_*\ t \Er p_*\ t' : T'$.
    \end{itemize}
    And in both cases, we have to converse using $p\inv$.
  \end{lemma}

  \begin{proof}
    Simultaneously by induction on $A \in s$ and $T \in s$.
    We only show the ``if'' direction.

    We show the case of functions, all the others are similar.

    \begin{caselist}
      \nextcase
      \begin{mathc}
        \rux{\Gamma \der \hideq{p} U \hEr U' : s \qquad
              \Gamma, x:U \der \hideq{q} V \hEr V'[p_*\ x / x] : s'
            }{\Gamma \der \hideq{\pi\ p\ q} \Pi x:U.V \Er \Pi x:U'.V' : s''
            }{(s,s',s'')}
      \end{mathc}
      \begin{mathc}
        \ru{\Gamma \der e : \Id (\Pi x:U.V)\ f\ f'
          }{\Gamma \der \hideq{e} f \Er f' : \Pi x:U.V}
      \end{mathc}
      Using $\ap (\pi\ p\ q)_*\ e$ we have the solution directly.
    \end{caselist}
  \end{proof}

  \begin{lemma}[Symmetry and Transitivity]
    \label{lem:s-per}
    Let $\Gamma \der \hideq{p} T \hEr T : s$.
    \leavevmode
    \begin{itemize}
      \item If $\Gamma \der \hideq{q} t \hEr t' : T$ then
      $\Gamma \der \hideq{q\inv} t' \hEr t : T$.
      \item If $\Gamma \der \hideq{q_1} t_1 \hEr t_2 : T$ and
      $\Gamma \der \hideq{q_2} t_2 \hEr t_3 : T$
      then $\Gamma \der \hideq{q_1 \cons q_2} t_1 \hEr t_3 : T$.
    \end{itemize}
  \end{lemma}

  \begin{proof}
    We generalize the two statements to whnfs
    $\Gamma \der \hideq{p} A \Er A : s$ and
    prove all four statements simultaneously by induction.

    \begin{center}
    \(
      \ru{\Gamma \der p : \Id s\ T\ T
        }{\Gamma \der \hideq{\squash{p}} \squash{T} \Er \squash{T} : s'}
    \)
    \end{center}

    \begin{caselist}
      \nextcase Symmetry.
      \begin{mathc}
        \ru{\Gamma \der e : \Id \squash{T}\ f\ f'
          }{\Gamma \der \hideq{e} f \Er f' : \squash{T}}
      \end{mathc}
      Symmetry holds directly with $e\inv$.

      \nextcase Transitivity also holds immediately with $e_1 \cons e_2$.
    \end{caselist}
  \end{proof}

  \begin{lemma}[Into the logical relation]
    Let $T \in s$.
    If $\Gamma \der e : \Id T\ n\ n'$ then $\Gamma \der n \hEr n' : T$.
  \end{lemma}

  \begin{proof}
    This holds directly by definition of $\hEr$.
  \end{proof}

  \paradot{Validity in the Model}

  We define a ``typing'' judgement for substitutions by induction on the
  destination context.
  %
  \begin{center}
  \(
    \ru{}{\Delta \der \sigma : \cdot}
    \qquad
    \ru{\Delta \der \sigma : \Gamma \qquad
        \Delta \der \sigma(x) : U \sigma
      }{\Delta \der \sigma : \Gamma, x:U}
  \)
  \end{center}

  We then define the context ($\Der \Gamma$), type ($\Gamma \Der T = T'$) and
  term ($\Gamma \Der t = t' : T$) validity relations by induction on the length
  of contexts.

  \begin{mathc}
    \ru{}{\Der \cdot}
    \qquad
    \ru{\Der \Gamma \qquad
        \Gamma \Der U
      }{\Der \Gamma, x:U}
    \qquad
    \ru{\Gamma \Der T = T' : s
      }{\Gamma \Der T = T'}
    \qquad
    \ru{\Gamma \Der T = T
      }{\Gamma \Der T}
  \end{mathc}

  \begin{mathc}
    \dru{\Der \Gamma \qquad
         (\Gamma \Der T \text{ unless } T = s)
       }{\forall \Delta, \sigma,\ %
         \Delta \der \sigma : \Gamma \gives
         \Delta \der t \sigma \hEr t' \sigma : T \sigma
       }{\Gamma \Der t = t' : T}
    \qquad
    \ru{\Gamma \Der t = t : T
      }{\Gamma \Der t : T}
  \end{mathc}

  \begin{lemma}[Validity is a PER]
    The relation $\Gamma \Der \_ = \_ : T$ is symmetric and transitive.
  \end{lemma}

  \begin{proof}
    \leavevmode
    \begin{caselist}
      \nextcase Symmetry.\\
      Assume $\Gamma \Der t = t' : T$ and show $\Gamma \Der t' = t : T$.
      Now assume $\Delta \der \sigma : \Gamma$ and show
      $\Gamma \der t' \sigma \hEr t \sigma : T \sigma$ (the other goals are
      already assumed). This holds because of lemma~\ref{lem:s-per} by
      instantiating the hypothesis with $\sigma$.

      \nextcase Transitivity.\\
      Assume $\Gamma \Der t_1 = t_2 : T$ and $\Gamma \Der t_2 = t_3 : T$ and
      show $\Gamma \Der t_1 = t_3 : T$.
      Now assume $\Delta \der \sigma : \Gamma$ and show
      $\Gamma \der t_1 \sigma \hEr t_3 \sigma : T \sigma$.
      We instantiate the hypotheses with $\sigma$ and conclude by transitivity
      of $\hEr$ (lemma~\ref{lem:s-per}).
    \end{caselist}
  \end{proof}

  \begin{lemma}[Function type injectivity is valid]
    \label{lem:fun-inj-valid}
    If $\Gamma \Der U \to T = U' \to T'$ then $\Gamma \Der U = U'$ and
    $\Gamma, x:U \Der T = T'$.
  \end{lemma}

  \begin{proof}
    Assume arbitrary $\Delta \der \sigma : \Gamma$.
    We have
    $\Delta \der U \sigma \to T\sigma \hEr U'\sigma \to T' \sigma : s_3$
    for some $s_3$. Thus, by definition,
    $\Delta \der \hideq{p} U \sigma \hEr U' \sigma : s_1$ for some $s_1$ and
    $p$. So, $\Gamma \Der U = U'$.

    We also have
    $\Delta, x:U\sigma \der T \sigma \hEr T'\sigma : s_2$ for some $s_2$
    (as $T'$ doesn't depend on $x$).
    Thus, for any $\rho$ that extends $\sigma$ on $x$, we have the result,
    enough to conclude $\Gamma, x:U \Der T = T'$.
  \end{proof}

  \begin{lemma}[Context Satisfiable]
    If $\Der \Gamma$ then $\der \Gamma$ and $\Gamma \der \id : \Gamma$.
  \end{lemma}

  \begin{proof}
    By induction on $\Gamma$.
  \end{proof}

  \paradot{Fundamental Theorem}
  We prove a series of lemmata which constitute parts of the fundamental theorem
  for the Kripke logical relation.

  Although, we state them with $\Gamma \Der \mathcal{J}$ as hypothesis,
  it is only to explicit the induction hypothesis we have from
  $\Gamma \der \mathcal{J}$, thus any hypothesis consisting of a conversion
  is preserved (because we use the same substitution on both sides),
  meaning the expressions are still well-typed.

  \begin{lemma}[Validity of $\beta$-reduction]
    \leavevmode
    \begin{mathc}
      \ru{\Gamma, x:U \Der t : T \qquad
          \Gamma \Der u : U
        }{\Gamma \Der (\lambda x:U.t)\ u = t[u/x] : T[u/x]}
    \end{mathc}
  \end{lemma}

  \begin{proof}
    $\Der \Gamma$ is contained in $\Gamma \Der u : U$.
    Now, given $\Delta \der \rho : \Gamma$, we need to show
    $\Delta \der (\lambda x:U.t)\rho\ u\rho \hEr (t[u/x])\rho : (T[u/x])\rho$
    and $\Delta \der (T[u/x])\rho \hEr (T[u/x])\rho : s$ for some $s$
    (to get $\Gamma \Der T[u/x]$)

    Let $\sigma = (\rho, u \rho/x)$.
    So $\Delta \der \sigma : \Gamma, x:U$.
    Thus, $\Delta \der t \sigma \hEr t' \sigma : T\sigma$
    \ie $\Delta \der (t[u/x])\rho \hEr (t[u/x])\rho : (T[u/x])\rho$
    which is weak-head convertible to our first goal.
    The first hypothesis, instantiated with $\sigma$ also gives
    $\Gamma \Der T[u/x]$.
  \end{proof}

  \begin{lemma}[Validity of $\eta$]
    \leavevmode
    \begin{mathc}
      \ru{\Gamma \Der t : \Pi x:U.T
        }{\Gamma \Der t = \lambda x:U.t\ x : \Pi x:U.T}
    \end{mathc}
  \end{lemma}
  Similar.

  \begin{lemma}[Validity of function equality]
    \leavevmode
    \begin{mathc}
      \ru{\Gamma \Der U = U' \qquad
          \Gamma, x:U \Der t = t' : T
        }{\Gamma \Der \lambda x:U.t = \lambda x:U'.t' : \Pi x:U.T}
    \end{mathc}
  \end{lemma}
  Ok.

  \begin{lemma}[Validity of $\elim$ reduction]
    \leavevmode
    \begin{mathc}
      \ru{\Gamma \Der f : U \to T \qquad
          \Gamma \Der h : \Pi_{x:T} \Pi_{y:T} \Id T\ x\ y \qquad
          \Gamma \Der t : U \qquad
          T \nequiv s
        }{\Gamma \Der \elim(U,T,f,h,[t]) = f\ t : T}
    \end{mathc}
  \end{lemma}

  \begin{proof}
    $\Der \Gamma$ is contained in any of the hypothesese.
    $\Gamma \Der T$ is implied by $\Gamma \Der f : U \to T$ and
    lemma~\ref{lem:fun-inj-valid}.
    %
    Now assume $\Delta, \sigma$ such that
    $\Delta \der \sigma : \Gamma$ and
    show (for some $p$)
    \[\Delta \der \hideq{p}
    \elim(U \sigma,T \sigma,f \sigma,h \sigma,[t \sigma]) \hEr
    f \sigma\ t \sigma : T \sigma.\]
    This reduces (by weak-head conversion) to showing
    $\Delta \der \hideq{p}
    f \sigma\ t \sigma \hEr
    f \sigma\ t \sigma : T \sigma$.
    (We cannot conclude yet, because $T \sigma$ might be a sort).

    We instantiate the first hypothesis to get
    $\Delta \der \hideq{q} f \sigma \hEr f \sigma : U \sigma \to T \sigma$
    and the last to get
    $\Delta \der \hideq{r} t \sigma \hEr t \sigma : U \sigma$
    for some $q,r$.
    We then get
    $\Delta \der \hideq{p} f \sigma\ t \sigma \hEr
    f \sigma\ t \sigma : T \sigma$
    where $p$ is described from $q$ and $r$ in the agda file ($elim-beta$).
  \end{proof}

  \begin{lemma}[Validity of $\elim$ equality]
    \leavevmode
    \begin{mathc}
      \dru{\Gamma \Der U = U' \qquad
           \Gamma \Der T = T' \qquad
           T, T' \nequiv s
         }{\Gamma \Der f = f' : U \to T \qquad
           \Gamma \Der h, h' : \Pi_{x:T} \Pi_{y:T} \Id T\ x\ y \qquad
           \Gamma \Der t,t' : \squash{U}
         }{\Gamma \Der \elim(U,T,f,h,t) = \elim(U',T',f',h',t') : T}
    \end{mathc}
  \end{lemma}

  \begin{proof}
    $\Der \Gamma$ is contained in any of the hypotheses.
    $\Gamma \Der T$ follows from $\Gamma \Der T = T'$ and the fact that validity
    is a PER.
    Now assume $\Delta, \sigma$ such that
    $\Delta \der \sigma : \Gamma$ and
    show
    \[\Delta \der \elim(U \sigma,T \sigma,f \sigma,h \sigma,t \sigma) \hEr
    \elim(U' \sigma,T' \sigma,f' \sigma,h' \sigma,t' \sigma) : T \sigma.\]
    %
    We look at $a$ and $a'$ the whnfs of $t \sigma$ and $t' \sigma$.
    We have three possible cases (by symmetry):
    \begin{caselist}
      \nextcase $a \equiv [u]$ and $a' \equiv [u']$.
      Thus, our goal becomes
      $\Delta \der f \sigma\ u \hEr f' \sigma\ u' : T \sigma$.
      Since $T \nequiv s$ is stated positively, $T \sigma$ is not a
      sort. So we only need a proof of equality which is deduced from $h$.

      \nextcase $a \equiv n$ and $a' \equiv [u']$.
      For similar reasons, this holds (we can still instantiate $h$ on the
      whnfs).

      \nextcase $a \equiv n$ and $a' \equiv n'$.
      Likewise.
    \end{caselist}
  \end{proof}

  \begin{lemma}[Validity of $\seq$]
    \leavevmode
    \begin{mathc}
      \ru{\Gamma \Der T = T' : s \qquad
          \Gamma \Der t = t' : \squash{T} \qquad
          \Gamma \Der u = u' : \squash{T}
        }{\Gamma \Der \seq(T,t,u) = \seq(T',t',u') : \Id \squash{T}\ t\ u}
    \end{mathc}
  \end{lemma}

  \begin{proof}
    $\Der \Gamma$ is trivial again.
    $\Der \Id \squash{T}\ t\ u$ is deduced from all the hypotheses combined
    with the fact that validity is a PER.
    %
    Now assume $\Delta, \sigma$ such that
    $\Delta \der \sigma : \Gamma$ and show
    $\Delta \der \seq(T\sigma,t\sigma,u\sigma) \hEr
    \seq(T'\sigma,t'\sigma,u'\sigma) : \Id \squash{T\sigma}\ t\sigma\ u\sigma$.

    Since, $\Id \squash{T\sigma}\ t\sigma\ u\sigma$ is positively not a sort,
    we only need to prove the equality in the system, and $\refl$ works
    (if we refer to our original hypotheses, and not the induction ones).
  \end{proof}

  \begin{theorem}[Fundamental theorem of logical relations]
    \leavevmode
    \begin{itemize}
      \item If $\der \Gamma$ then $\Der \Gamma$.
      \item If $\Gamma \der t : T$ then $\Gamma \Der t : T$.
      \item If $\Gamma \der t = t' : T$ then $\Gamma \Der t = t' : T$.
    \end{itemize}
  \end{theorem}

  \begin{proof}
    By induction on the derivation.
  \end{proof}

  \begin{corollary}[Syntactic validity]
    \leavevmode
    \begin{itemize}
      \item If $\Gamma \der t : T$ then $\Gamma \der T$.
      \item If $\Gamma \der t = t' : T$ then $\Gamma \der t : T$ and
      $\Gamma \der t' : T$.
    \end{itemize}
  \end{corollary}
  Alright.

  \section{Meta-Theoretic Consequences of the Model Construction}

  \paradot{Inversion}

  \begin{lemma}[Inversion]
    \label{lem:inversion}
    \leavevmode
    \begin{itemize}
      \item If $\Gamma \der x : T$ then $(x:U) \in \Gamma$ for some
      $U$ with $\Gamma \der U \le T$.
      \item If $\Gamma \der \lambda x:U.t : T$ then $\Gamma, x:U \der t : T'$
      for some $T'$ with $\Gamma \der \Pi x:U.T' \le T$.
      \item If $\Gamma \der t\ u : T$ then $\Gamma \der t : \Pi x:U.T'$
      for some $U$ and $T'$ with $\Gamma \der T'[u/x] \le T$.
      \item If $s : T$ then there is $s'$ such that $(s,s') \in \Ax$
      and $\Gamma \der s' = T$.
      \item If $\Gamma \der \Pi x:U.T' : T$ then $\Gamma \der U : s$
      and $\Gamma, x:U \der T : s'$ and for some $s''$ we have
      $\Gamma \der s'' = T$ and $(s,s',s'') \in \R$.
      \item If $\Gamma \der \refl_t : T$ then $\Gamma \der t : U$
      for some $U$ with $\Gamma \der \Id U\ t\ t = T$.
      \item If $\Gamma \der \J(A,C,b,u,v,p) : T$ then $\Gamma \der A : s$
      and $\Gamma \der C : \Pi x:A. \Pi y:A. (\Id A\ x\ y) \to s'$ and
      $\Gamma \der b : \Pi x:A. C\ x\ x\ \refl_x$ and $\Gamma \der u, v : A$
      and $\Gamma \der p : \Id A\ u\ v$ for some $s,s'$ with
      $\Gamma \der C\ u\ v\ p \le T$.
      \item If $\Gamma \der \Id T'\ t\ t' : T$ then $\Gamma \der t,t' : T'$
      and $\Gamma \der s = T$ for some $s$.
      \item If $\Gamma \der [t] : T$ then $\Gamma \der t : T'$ for some
      $T'$ with $\Gamma \der \squash{T'} = T$.
      \item If $\Gamma \der \elim(A,B,f,h,a) : T$ then
      $\Gamma \der f : A \to B$ and
      $\Gamma \der h : \Pi_{x:A} \Pi_{y:A} \Id A\ x\ y$
      and $\Gamma \der a : \squash{A}$ and $\Gamma \der B \le T$.
      \item If $\Gamma \der \seq(T',u,v) : T$ then $\Gamma \der T' : s$ for
      some $s$ and $\Gamma \der u, v : \squash{T'}$ and
      $\Gamma \der T = \Id \squash{T'}\ u\ v$.
      \item If $\Gamma \der \squash{T'} : T$ then $\Gamma \der T' : s$ for
      some $s$ and $\Gamma \der s' = T'$ for some $s'$.
    \end{itemize}
  \end{lemma}

  \begin{proof}
    By induction on the typing derivation.
  \end{proof}

  \paradot{Normalization and subject reduction}

  We assume our system to be normalizing with the subject reduction property.

  \paradot{Consistency}

  Importantly, not every type is inhabited, thus, it can be used as a logic.
  A prerequisite is that types can be distinguished, which follows immediately
  from the construction of the logical relation.

  \begin{lemma}[Type constructor discrimination]
    \label{lem:cons-discr}
    Neutral types, sorts, $\Pi$-types, identity types and
    squash types are mutually unequal.
    \leavevmode
    \begin{itemize}
      \item $\Gamma \der N \neq s$.
      \item $\Gamma \der N \neq \Pi x:U.T$.
      \item $\Gamma \der N \neq \Id T\ u\ v$.
      \item $\Gamma \der N \neq \squash{T}$.
      \item $\Gamma \der s = s'$ implies $\Id s s'$ is inhabited in $\Gamma$.
      \item $\Gamma \der s \neq \Pi x:U.T$.
      \item $\Gamma \der s \neq \Id T\ u\ v$.
      \item $\Gamma \der s \neq \squash{T}$.
      \item $\Gamma \der \Pi x:U.T \neq \Id T'\ u\ v$.
      \item $\Gamma \der \Pi x:U.T \neq \squash{T'}$.
      \item $\Gamma \der \Id T\ u\ v \neq \squash{T'}$.
    \end{itemize}
  \end{lemma}

  \begin{proof}
    By the fundamental theorem applied to the identity substitution.
    For instance, assume $\Gamma \der N = s$, by fundamental theorem,
    $\Gamma \Der N = s$, so, $\Gamma \der \hideq{p} N \Er s$, since this is
    comparision of types, there is no rule to derive it, hence the
    contradiction.
  \end{proof}

  \begin{theorem}[Consistency]
    $X : \Type_0 \nvdash t : X$.
  \end{theorem}

  \begin{proof}
    Let $\Gamma = (X : \Type_0)$.
    Assuming $\Gamma \der t : X$, we have $\Gamma \der a : X$ for $a$ the whnf
    of $t$. We invert on the typing of $a$.
    By lemma~\ref{lem:cons-discr}, $X$ cannot be equal (or in a subtype
    relation since it preserves the shape) to a sort, a $\Pi$-type,
    an indentity type or a squash type.
    Thus, $a$ can neither be a $\lambda$, nor a $\refl$,
    nor a $[\_]$ nor a $\Pi$-type, an indentity type or
    a squash type, nor a sort, it can only be neutral.

    $X$ being the only variable, it is the only reason for the term to be
    blocked.
    Since $X$ is not of $\Pi$-type, it cannot be applied.
    Since $X$ is not of identity type, it cannot be eliminated with $\J$.
    Since $X$ is not of squash type, it cannot be eliminated through $\elim$.
    So $a \equiv X$ and then $\Gamma \der X : X$, implying
    $\Gamma \der \Type_0 \le X$ by inversion
    (lemma~\ref{lem:inversion}), meaning $\Gamma \der \Type_i = X$ for some
    $i$ which is in contradiction with lemma~\ref{lem:cons-discr}.
  \end{proof}

\end{document}
