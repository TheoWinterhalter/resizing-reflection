\documentclass[a4paper,english]{lipics-utf8x}

\usepackage[T1]{fontenc} %

\usepackage{amsmath, amssymb, amsfonts, stmaryrd}
\usepackage{bm}
\usepackage{pifont}
\PrerenderUnicode{é} % For the author names in the heading

% Add some colors
\usepackage[usenames,dvipsnames,svgnames,table]{xcolor}
\usepackage{hyperref}
\hypersetup{
 linktocpage,
 colorlinks,
 citecolor=BlueViolet,
 filecolor=red,
 linkcolor=Blue,
 urlcolor=BrickRed
}

\usepackage{graphicx}
\usepackage{placeins}

% Meta comment
\newcommand\meta[1]{\noindent\textcolor{blue}{\emph{#1}}}

% Include the macro file
% evergreens
\newcommand{\der}{\,\vdash}
\newcommand{\Der}{\,\Vdash}

% semantic brackets
\def\lv{\mathopen{{[\kern-0.14em[}}}    % opening [[ value delimiter
\def\rv{\mathclose{{]\kern-0.14em]}}}   % closing ]] value delimiter
\newcommand{\den}[1]{\lv #1 \rv}
\newcommand{\Den}[3][]{\den{#2}^{#1}_{#3}}
\newcommand{\dent}[2]{\llparenthesis#1\rrparenthesis_{#2}}

% latin etc. abbrev
\newcommand{\abbrev}[1]{#1} % alternative: \emph{#1}
\newcommand{\cf}{\abbrev{cf.}\ }
\newcommand{\eg}{\abbrev{e.\,g.}}
\newcommand{\Eg}{\abbrev{E.\,g.}}
\newcommand{\ie}{\abbrev{i.\,e.}}
\newcommand{\Ie}{\abbrev{I.\,e.}}
\newcommand{\etal}{\abbrev{et.\,al.}}
\newcommand{\wwlog}{w.\,l.\,o.\,g.} % \wlog is ``write into log file''
\newcommand{\Wlog}{W.\,l.\,o.\,g.}
\newcommand{\wrt}{w.\,r.\,t.}

% Inference rules
\newcommand{\rulename}[1]{\ensuremath{\mbox{\sc#1}}}
\newcommand{\ru}[2]{\dfrac{\begin{array}[b]{@{}c@{}} #1 \end{array}}{#2}}
\newcommand{\rux}[3]{\ru{#1}{#2}~#3}
\newcommand{\nru}[3]{#1\ \ru{#2}{#3}}
\newcommand{\nrux}[4]{#1\ \ru{#2}{#3}\ #4}
\newcommand{\dstack}[2]{\begin{array}[b]{c}#1\\#2\end{array}}
\newcommand{\dru}[3]{\ru{\dstack{#1}{#2}}{#3}}
\newcommand{\tru}[4]{\dru{\dstack{#1}{#2}}{#3}{#4}}
\newcommand{\trux}[5]{\dru{\dstack{#1}{#2}}{#3}{#4}\ #5}
\newcommand{\qru}[5]{\tru{\dstack{#1}{#2}}{#3}{#4}{#5}}
\newcommand{\ndru}[4]{#1\ \ru{\dstack{#2}{#3}}{#4}}
\newcommand{\ndrux}[5]{#1\ \ru{\dstack{#2}{#3}}{#4}\ #5}

% Symbols and names
\newcommand\Type{\operatorname{Type}}
\newcommand\isnType[2]{\operatorname{is-}#1\operatorname{-Type}\ #2}
\newcommand\nType[1]{#1\operatorname{-Type}}
\newcommand\R{\operatorname{R}}
\newcommand\emb[2]{\operatorname{embedding}#1\ #2}
\newcommand\RRe[2]{\operatorname{RR_e}#1\ #2}
% \newcommand\type{\ \bm{\operatorname{type}}}
\DeclareMathOperator{\type}{\ \mathbf{type}}
\DeclareMathOperator{\ctr}{\mathbf{ctr}}
\DeclareMathOperator{\refl}{\mathbf{refl}}
\newcommand\rew{\searrow}
\newcommand\gettype{\operatorname{.type}}
\newcommand\getproof{\operatorname{.proof}}
\newcommand\Var{\operatorname{Var}}
\newcommand\Exp{\operatorname{Exp}}
\newcommand\Ctx{\operatorname{Ctx}}
\newcommand\Whnf{\operatorname{Whnf}}
\newcommand\Wne{\operatorname{Wne}}


% Title and so...
\title{Working Resizing Rules under Univalence}
\author[1]{Théo Winterhalter}

\begin{document}

  \maketitle

  \begin{abstract}
    We show how to turn an equality into a strict equivalence. This is essential
    to instrumentalize resizing rules under the assumption of univalence.
    In its globality, our main result is a proof of consistency of resizing
    rules that relies on univalence and possibly the law of excluded middle
    (depending on which resizing rules we seek). The implementation, however,
    does not require the user to assume them. In a sense, they could be weaker
    versions of the univalence axiom (and excluded middle).
  \end{abstract}

  \section{Resizing of Equality}

  \subsection{Systems}

  This is the syntax of the system in which we start, which we will call
  $\Se$.

  \[
    \begin{array}{l@{~}l@{~}l@{~}r@{~}l@{\quad}l}
      \Var  & \ni & x,y,X,Y \\
      \Sort & \ni & s             & ::= & \Type_k \mbox{ }
                                                (k \in \mathbb{N}) \\
      \Exp  & \ni & t,u,T,U & ::= & s \mid \Pi x:U.T \mid
                                    \Id T\ t\ u \\
                         &&& \mid & x \mid \lambda x:U.t \mid t~u
                               \mid \refl(T,t) \mid
                               \J (T,U,t_{refl},u_1,u_2,t_{eq}) \\
      \Ctx  & \ni & \Gamma  & ::= & \cdot \mid \Gamma, x:T \\
    \end{array}
  \]
  %
  We assume it contains $\mA$ and $\mB$ that are equivalent (and respectively in
  universes $\sA$ and $\sB$), but up to univalence, we only assume that
  there is $\me$ such that $\der \me : \Id \sR\ \mA\ \mB$ for some $\sR$
  (note: we will only treat the case where this holds in the empty context,
  but it should scale easily).
  We will write $\eta$ and $\epsilon$ for the section and retraction (even
  though they can be explicited with $\me$).

  We then extend this system with $\mR$, $\inj(t)$ and $\proj(t)$ as well as
  the following rules (which we will note with $\derr$ instead of $\dere$)
  (this one will be called $\Sr$).

  \begin{mathc}
    \ru{\derr \Gamma
      }{\Gamma \derr \mR : \sB}
    \qquad
    \ru{\Gamma \derr t : \mA
      }{\Gamma \derr \inj(t) : \mR}
    \qquad
    \ru{\Gamma \derr t : \mR
      }{\Gamma \derr \proj(t) : \mA}
  \end{mathc}

  \begin{mathc}
    \ru{\Gamma \derr t : \mA
      }{\Gamma \derr \proj(\inj(t)) = t : \mA}
    \qquad
    \ru{\Gamma \derr t : \mR
      }{\Gamma \derr \inj(\proj(t)) = t : \mR}
  \end{mathc}

  % Finally, we define $\Sb$ which is an intermediary for the translation
  % with residuals for $\mR$ and $\me$ : $\mbB$ and $\mbe$.
  % This is simply an extension of $\Se$ but where
  % $\mbe : \Id s\ \mA\ \mbB$ and $\mbB$ behaves as $\mB$ but can be abstracted
  % over, meaning we can destruct $\mbe$ (basically, $\mbB$ is a variable
  % \emph{and} lives in universe $\sR$ instead of $\sB$, this implies that we
  % need to be careful when writing $\mbB$).
  % We will also assume we have $\eta^\bullet$ and $\epsilon^\bullet$ which are
  % basically $\eta$ and $\epsilon$ but on $\mbe$ instead of $\me$.

  \subsection{Translation for Consistency ($\Sr$ to $\Se$)}

  Because our translation is based on the derivation of a judgement and not
  only the judgement itself, we need to have some criterium of compatibility.
  It will also be helpful to ensure that the empty type is mapped to something
  close to the empty type (as in, it is empty as well).
  %
  Translated terms are paraletrized by $\mbA,\mbB,\mbe$ (in universe $\sR$)
  in order to make it possible to use $\J$ on them, as well as characterizing
  their behavior
  ``under it''. We extend this parametrization on contexts naturally.

  In the following $\Gamma \dere t = t'$ shall mean that there exists $A$
  such that $\Gamma \dere t = t' : A$ holds. We will write $t\prfl$ for
  $t[\mA,\mA,\refl]$, and $t\pfull$ for $t[\mA,\mB,\me]$.

  \begin{definition}[$\Gamma$-translation]
    For any parametrized context $\dere \Gamma$ we define the notion of
    $\Gamma$-translation of a term $u$ (in $\Se$), writing $\tr{t}{\Gamma}$ for
    the set of such terms, by induction on $u$.
    \begin{itemize}
      \item $t \in \tr{s}{\Gamma}$ iff
            $\Gamma\prfl \dere t\prfl = s$.
      \item $t \in \tr{\Pi x:A.B}{\Gamma}$ iff
            $\Gamma\prfl \dere t\prfl =
            \Pi x:A'\prfl.B'\prfl$ where
            $A' \in \tr{A}{\Gamma}$ and $B' \in \tr{B}{\Gamma, x:A'}$.
      \item $t \in \tr{\Id A\ u\ v}{\Gamma}$ iff
            $\Gamma\prfl \dere t\prfl = \Id A'\prfl\ u'\prfl\ %
            v'\prfl$ where $A' \in \tr{A}{\Gamma}, u' \in \tr{u}{\Gamma}$,
            $v' \in \tr{v}{\Gamma}$.
      \item $t \in \tr{x}{\Gamma}$ iff
            $\Gamma\prfl \dere t\prfl = x$.
      \item $t \in \tr{\lambda x:A.b}{\Gamma}$ iff
            $\Gamma\prfl \dere t\prfl = \lambda x:A'\prfl.b'\prfl$
            where $A' \in \tr{A}{\Gamma}$ and $b' \in \tr{b}{\Gamma, x:A'}$.
      \item $t \in \tr{a\ b}{\Gamma}$ iff
            $\Gamma\prfl \dere t\prfl = a'\prfl\ b'\prfl$ where
            $a' \in \tr{a}{\Gamma}, b' \in \tr{b}{\Gamma}$.
      \item $t \in \tr{\refl(A,u)}{\Gamma}$ iff
            $\Gamma\prfl \dere t\prfl = \refl(A'\prfl,u'\prfl)$ where
            $A' \in \tr{A}{\Gamma}, u' \in \tr{u}{\Gamma}$.
      \item \sloppy
            $t \in \tr{\J(A,C,b,u,v,p)}{\Gamma}$ iff
            $\Gamma\prfl \dere t\prfl = \J(A',C',b',u',v',p')\prfl$
            where $A',C',b',u',v',p'$ are $\Gamma$-translations of
            $A,C,b,u,v,p$ respectively.
      \item $t \in \tr{\mR}{\Gamma}$ iff
            $\Gamma\prfl \dere t\prfl = B'\prfl$ where
            $B' \in \tr{\mB}{\Gamma}$.
      \item $t \in \tr{\inj(u)}{\Gamma}$ iff
            $\Gamma\prfl \dere t\prfl = u'\prfl$ where
            $u' \in \tr{u}{\Gamma}$.
      \item $t \in \tr{\proj(u)}{\Gamma}$ iff
            $\Gamma\prfl \dere t\prfl = u'\prfl$ where
            $u' \in \tr{u}{\Gamma}$.
    \end{itemize}
  \end{definition}

  \begin{definition}[Context Translation]
    We define translations of a context $\Gamma$ by induction on $\Gamma$:
    \begin{itemize}
      \item $\cdot$ is a translation of $\cdot$.
      \item $\Gamma', x:A'$ is a translation of $\Gamma, x:A$ if $\Gamma'$ is
            a translation of $\Gamma$ and $A' \in \tr{A}{\Gamma'}$.
    \end{itemize}
  \end{definition}

  Before dealing with our theorem, we will look at an important lemma regarding
  the translation.

  \begin{lemma}[Equality of translations]
    \label{lem:transleq}
    If $t_1, t_2 \in \tr{u}{\Gamma}$ and if
    there exists $T$ such that
    $\Gamma\pfull \dere t_1\pfull, t_2\pfull : T\pfull$
    then there exists $h$ such that
    \[
      \Gamma\pfull \dere h\pfull : \Id T\pfull\ t_1\pfull\ t_2\pfull.
    \]
    %
    Besides, $\Gamma\prfl \dere h\prfl = \refl(T\prfl,t_1\prfl)$.
  \end{lemma}

  \begin{proof}
    Formally we prove by induction that
    $\Gamma\prfl \dere t_1\prfl = t_2\prfl$
    meaning $\Gamma\prfl \dere t\prfl = t'\prfl : T\prfl$
    (to do so we rely on unicity of typing).
    Then we have
    \[
      \Gamma\pfull \dere \J(\sR, P, p, \mA, \mB, \me)\ %
      (\mid \Gamma\pfull \mid) : \Id T\pfull\ t_1\pfull\ t_2\pfull
    \]
    where
    \[
      \begin{array}{l@{~}l}
        P &:= \lambda \mA,\mB,\me. \Pi \Gamma\pfull. %
              \Id T\pfull\ t_1\pfull\ t_2\pfull \\
        p &:= \lambda \mA. \lambda \Gamma\prfl. \refl
      \end{array}
    \]
    and
    \[
      \begin{array}{l@{~}l@{\qquad}l@{~}l}
        \Pi (\Gamma, x:T).T' &:= \Pi \Gamma. \Pi x:T. T' &
        \Pi \cdot. T &:= T \\
        \lambda (\Gamma, x:T).t &:= \lambda \Gamma. \lambda x:T.t &
        \lambda \cdot.t &:= t
      \end{array}
    \]
    and where $\mid \Gamma \mid$ means applying every variable in $\dom \Gamma$
    successively.
    We indeed have that it is convertible to $\refl$ when $\me$ is.
  \end{proof}

  \begin{lemma}[Translation of Convertibility]
    If $\Gamma \derr t = u : T$ and $\Gamma'$ is a translation of $\Gamma$ and
    $t' \in \tr{t}{\Gamma'}$ and $u' \in \tr{u}{\Gamma'}$ and
    $T' \in \tr{T}{\Gamma'}$ then
    $\Gamma'\prfl \dere t'\prfl = u'\prfl : T'\prfl$.
  \end{lemma}

  \noindent
  This lemma allows us to consider $\refl$ on such terms.
  \meta{Prove it!}


  We can now go on defining the theorem using the above definitions.
  \meta{Maybe we can make it more precise? Say it holds for $\prfl$ and $\pfull$
  instead of $\pany$?}

  \begin{theorem}[Translation]
    \label{thm:transl}
    \leavevmode
    \begin{enumerate}
      \item If $\Gamma \derr t : T$ then there exists $\Gamma'$ a well-formed
      translation of $\Gamma$ and for any such $\Gamma'$, there are
      $t' \in \tr{t}{\Gamma}$ and $T' \in \tr{T}{\Gamma}$
      such that $\Gamma'\pany \dere t'\pany : T'\pany$.
      \item If $\Gamma \derr t = u : T$ then there exists $\Gamma'$ a
      well-formed translation of $\Gamma$ and for any such $\Gamma'$, there are
      $t' \in \tr{t}{\Gamma}$ and $u' \in \tr{u}{\Gamma}$ and
      $T' \in \tr{T}{\Gamma}$ and $h$ such that
      $\Gamma'\pany \dere h\pany : \Id T'\pany\ t'\pany\ u'\pany$ and
      $\Gamma'\prfl \dere h\prfl = \refl$.
      \item If $\derr \Gamma$ then there exists $\Gamma'$ a translation of
      $\Gamma$ such that $\dere \Gamma'$.
    \end{enumerate}
  \end{theorem}

  \begin{proof}
    By induction on the derivation (we grayed out the cases that we deem
    simple or merely a reproduction of another case, meaning that you shouldn't
    have to focus on them if you want to get an idea of where the proof lies).

    \leavevmode
    \begin{caselist}
      \begin{graycase}
        \begin{mathc}
          \ru{}{\derr \cdot}
        \end{mathc}
        We have $\dere \cdot$, which corresponds to a valid translation.
      \end{graycase}

      \begin{graycase}
        \begin{mathc}
          \rux{\Gamma \derr A : s
             }{\derr \Gamma, x ; A
             }{x \notin \dom \Gamma}
        \end{mathc}
        By induction hypothesis, we have $\Gamma' \dere A' : t$ a valid
        translation\footnote{This stands for
        $\Gamma'\pany \dere A'\pany : t\pany$}.
        By lemma~\ref{lem:transleq}, we have
        $\Gamma' \dere A'' : s$ which is still a valid translation
        (indeed $s \in \tr{s}{\Gamma}$, it parameterized in the sense that it
        does not mention $\mA,\mB,\me$).
        So $\dere \Gamma', x : A''$ and $\Gamma', x : A''$ is a translation of
        $\Gamma, x : A$.
        (Note: Being a translation means that $x \notin \dom \Gamma$ ensures
        $x \notin \dom \Gamma'$).
      \end{graycase}

      \begin{graycase}
        \begin{mathc}
          \rux{\derr \Gamma
             }{\Gamma \derr s_1 : s_2
             }{(s_1,s_2)}
        \end{mathc}
        By induction hypothesis, $\dere \Gamma'$ is a translation.
        And so is $\Gamma' \dere s_1 : s_2$.
      \end{graycase}

      \begin{graycase}
        \begin{mathc}
          \rux{\derr \Gamma
             }{\Gamma \derr x : A
             }{(x : A) \in \Gamma}
        \end{mathc}
        By induction hypothesis, $\dere \Gamma'$ is a translation and
        $(x : A') \in \Gamma'$ for $A'$ a translation of $A$.
        Thus $\Gamma' \dere x : A'$ is a valid translation.
      \end{graycase}

      \begin{graycase}
        \begin{mathc}
          \rux{\Gamma \derr A : s_1 \qquad
               \Gamma, x : A \derr B : s_2
             }{\Gamma \derr \Pi x:A.B : s_3
             }{(s_1,s_2,s_3)}
        \end{mathc}
        By induction hypothesis, $\Gamma' \dere A' : t_1$ is a valid
        translation.
        By lemma~\ref{lem:transleq}, $\Gamma' \dere A' : s_1$ (up to transport).
        Thus, $\dere \Gamma', x : A'$, and so, by second induction hypothesis,
        $\Gamma', x:A' \dere B' : t_2$, but likewise, we assume we have
        $\Gamma', x:A' \dere B' : s_2$ instead.
        So $\Gamma' \dere \Pi x:A'.B' : s_3$ is a valid translation.
      \end{graycase}

      \begin{graycase}
        \begin{mathc}
          \rux{\Gamma \derr A : s_1 \qquad
               \Gamma, x:A \derr b : B : s_2
             }{\Gamma \derr \lambda x:A.b : \Pi x:A.B
             }{(s_1,s_2,s_3)}
        \end{mathc}
        By induction hypotheses, $\Gamma' \dere A' : s_1$ (implicitely using
        lemma~\ref{lem:transleq}) and $\Gamma', x:A' \dere b' : B''$ and
        $\Gamma', x:A' \dere B' : s_2$.
        Using a transport\footnote{It is alright to assume that $B'$ and $B''$
        can have the same type (to use lemma~\ref{lem:transleq})
        thanks to cumulativity, combined with the fact that being on the right
        of colons implies you can be typed by a sort.},
        we can assume $\Gamma', x:A' \dere b' : B'$ instead.
        Now, $\Gamma' \dere \lambda x:A'.b' : \Pi x:A'.B'$.
      \end{graycase}

      \nextcase
      \begin{mathc}
        \ru{\Gamma \derr F : \Pi x:A.B \qquad
            \Gamma \derr a : A
          }{\Gamma \derr F\ a : B[x := a]}
      \end{mathc}
      By induction hypotheses, $\Gamma' \dere F' : T$ and
      $\Gamma' \dere a' : A''$.
      Since $T \in \tr{\Pi x:A.B}{\Gamma'}$, by definition, there exist
      $A' \in \tr{A}{\Gamma'}$ and $B' \in \tr{B}{\Gamma', x:A'}$.
      Thus, $\Pi x:A'.B' \in \tr{\Pi x:A.B}{\Gamma}$ and, by
      lemma~\ref{lem:transleq}, we can safely assume
      $\Gamma' \dere F' : \Pi x:A'.B'$. We can also assume
      $\Gamma' \dere a' : A'$.
      Then, we finally have $\Gamma' \dere F'\ a' : B'[x := a']$.
      \meta{Perhaps should we have a lemma to justify translations are
      preserved under substitutions.}

      \begin{graycase}
        \begin{mathc}
          \ru{\Gamma \derr A : s \qquad
              \Gamma \derr u : A \qquad
              \Gamma \derr v : A
            }{\Gamma \derr \Id A\ u\ v : s}
        \end{mathc}
        By induction hypotheses, $\Gamma' \dere A' : s$ (using the trick
        aforementioned), $\Gamma' \dere u' : A_u$ and $\Gamma' \dere v' : A_v$,
        using the same trick again, we assume $\Gamma' \dere u' : A'$ and
        $\Gamma' \dere v' : A'$.
        Thus, $\Gamma' \dere \Id A'\ u'\ v' : s$.
      \end{graycase}

      \begin{graycase}
        \begin{mathc}
          \ru{\Gamma \derr A : s \qquad
              \Gamma \derr u : A
            }{\Gamma \derr \refl(A,u) : \Id A\ u\ u}
        \end{mathc}
        By induction hypothesis, $\Gamma' \dere A' : s$ and
        $\Gamma' \dere u' : A'$ so
        $\Gamma' \dere \refl(A',u') : \Id A'\ u'\ u'$.
      \end{graycase}

      \begin{graycase}
        \begin{mathc}
          \tru{\Gamma \derr A : s \qquad
               \Gamma \derr C : \Pi x:A. \Pi y:A. (\Id A\ x\ y) \to s'
             }{\Gamma \derr b : \Pi x:A. C\ x\ x\ \refl(A,x)
             }{\Gamma \derr u, v : A \qquad
               \Gamma \derr p : \Id A\ u\ v
             }{\Gamma \derr \J (A,C,b,u,v,p) : C\ u\ v\ p}
        \end{mathc}
        By induction hypotheses, $\Gamma' \dere A' : s$ and
        $\Gamma' \dere C' : \Pi x:A'. \Pi y:A'. (\Id A'\ x\ y) \to s'$ and
        $\Gamma' \dere b' : \Pi x:A'. C'\ x\ x\ \refl(A',x)$ and
        $\Gamma' \dere u' : A'$ and $\Gamma' \dere v : A'$ and
        $\Gamma' \dere p' : \Id A'\ u'\ v'$ (the trick can be used as long as we
        have a valid translation, which we have in all of the cases above).
        Thus $\Gamma' \dere \J(A',C',b',u',v',p') : C'\ u'\ v'\ p'$.
      \end{graycase}

      \nextcase
      \begin{mathc}
        \ru{\Gamma \derr a : A \qquad
            \Gamma \derr A = B : s
          }{\Gamma \derr a : B}
      \end{mathc}
      By the first induction hypothesis, there are $\Gamma'$, $a'$ and $A'$
      such that $\Gamma' \dere a' : A'$ (and that are translations).
      Then, we can instantiate the second induction hypothesis with $\Gamma'$:
      we thus have $A'',B',t$ that are $\Gamma'$-translations of $A,B,s$ and
      $h$ such that $\Gamma' \dere h : \Id t\ A''\ B'$.
      By lemma~\ref{lem:transleq}, there is $h'$ such that
      $\Gamma' \dere h' : \Id s'\ A'\ A''$ for some $s'$.
      Then $\Gamma' \dere (h \cons h')_*\ a' : B'$ with everything being
      $\Gamma'$-translations.

      \begin{graycase}
        \begin{mathc}
          \ru{\Gamma \derr a : A
            }{\Gamma \derr a = a : A}
        \end{mathc}
        By induction hypothesis, $\Gamma' \dere a' : A'$ and thus
        $\Gamma' \dere \refl(A',a') : \Id A'\ a'\ a'$.
      \end{graycase}

      \begin{graycase}
        \begin{mathc}
          \ru{\Gamma \derr a_1 = a_2 : A
            }{\Gamma \derr a_2 = a_1 : A}
        \end{mathc}
        By induction hypothesis, $\Gamma' \dere h : \Id A'\ a'_1\ a'_2$.
        Thus $\Gamma' \dere h\inv : \Id A'\ a'_2\ a'_1$.
      \end{graycase}

      \begin{graycase}
        \begin{mathc}
          \ru{\Gamma \derr a_1 = a_2 : A \qquad
              \Gamma \derr a_2 = a_3 : A
            }{\Gamma \derr a_1 = a_3 : A}
        \end{mathc}
        By induction hyoptheses (and using the ``trick''),
        $\Gamma' \dere h_1 : \Id A'\ a'_1\ a'_2$ and
        $\Gamma' \dere h_2 : \Id A'\ a'_2\ a'_3$.
        So $\Gamma' \dere h_1 \cons h_2 : \Id A'\ a'_1\ a'_3$.
      \end{graycase}

      \begin{graycase}
        \begin{mathc}
          \rux{\Gamma \derr a : A : s_1 \qquad
               \Gamma, x : A \derr b : B : s_2
             }{\Gamma \derr (\lambda x:A.b)\ a = b[x := a] : B[x := a]
             }{(s_1,s_2,s_3)}
        \end{mathc}
        By induction hypotheses, $\Gamma' \dere a' : A' : s_1$ and
        $\Gamma', x:A' \dere b' : B' : s_2$, then
        $\Gamma' \dere \refl(B'[x := a'], b'[x := a']) : \Id B'[x := a']\ %
        ((\lambda x:A'.b')\ a')\ b'[x := a']$.
      \end{graycase}

      \begin{graycase}
        \begin{mathc}
          \rux{\Gamma \derr A_1 = A_2 : s_1 \qquad
               \Gamma, x : A_1 \derr B_1 = B_2 : s_2
             }{\Gamma \derr \Pi x:A_1.B_1 = \Pi x:A_2.B_2
             }{(s_1,s_2,s_3)}
        \end{mathc}
        By induction hypotheses, $\Gamma' \dere h_1 : \Id s_1\ A'_1\ A'_2$
        and $\Gamma', x:A'_1 \dere h_2 : \Id s_2\ B'_1\ B'_2$.
        Thus,
        $\Gamma' \dere h_3 : \Id s_3\ (\Pi x:A'_1.B'_1)\ (\Pi x:A'_2.B'_2)$
        for some $h_3$ that we won't state out explicitely.
      \end{graycase}

      \nextcase
      \begin{mathc}
        \rux{\Gamma \derr A_1 = A_2 : s_1 \qquad
             \Gamma, x : A_1 \derr b_1 = b_2 : B : s_2
           }{\Gamma \derr \lambda x:A_1.b_1 = \lambda x:A_2.b_2 : \Pi x:A_1.B
           }{(s_1,s_2,s_3)}
      \end{mathc}
      By induction hypothesis, there are $\Gamma',A'_1,A'_2,t_1$ translations
      and $h_1$ such that $\Gamma' \dere h_1 : \Id t_1\ A'_1\ A'_2$.
      Thus, $\Gamma' \dere A'_1 : t_1$, and by lemma~\ref{lem:transleq}
      (noting that any sort is a translation of itself (modulo injection))
      we have $e_1$ such that $\Gamma' \dere e_1 : \Id s_4\ t_1\ s_1$ and
      then $\Gamma' \dere (e_1)_*\ A'_1 : s_1$.
      $\Gamma', x : (e_1)_*\ A'_1$ is a well-formed translation of
      $\Gamma, x:A$. We will write it $\Delta$ for short.

      We can instantiate the second hypothesis to get
      $\Delta \dere h_2 : \Id B'\ b'_1\ b'_2$ and $\Delta \dere B'' : t_2$.
      By lemma~\ref{lem:transleq}, we have $e_2,e_3$ such that
      $\Delta \dere e_2 : \Id s_5\ t_2\ s_2$ and
      $\Delta \dere (e_3)_* : B' \to (e_2)_*\ B''$.

      The problem is that
      \[
        \lambda (x : (e_1)_*\ A'_2). (e_3)_*\ %
        b'_2[x := ((e_1)_* (h_1\inv))_*\ x]
      \]
      has type
      $\Pi (x : (e_1)_*\ A'_2). (e_2)_*\ B''$ instead of
      $\Pi (x : (e_1)_*\ A'_1). (e_2)_*\ B''$.
      However these two types are equal thanks to $h_1$.
      (We could explicit it, but everyone should be convinced. We can then
      transport the whole $\lambda$ expression, we keep the translation
      property and thus equality under $\prfl$).

      We finally have
      $\Gamma' \dere e_5 : \Id T\ u_1\ u_2$ for some $e_4,e_5$,
      where
      \[
        \begin{array}{l@{~}l}
          T   &:= \Pi (x : (e_1)_*\ A'_1). (e_2)_*\ B'' \\
          u_1 &:= \lambda (x : (e_1)_*\ A'_1). (e_3)_*\ b'_1 \\
          u_2 &:= (e_4)_*\ %
                   (\lambda (x : (e_1)_*\ A'_2).
                    (e_3)_*\ b'_2[x := ((e_1)_* (h_1\inv))_*\ x])
        \end{array}
      \]
      And everything behaves well under $\prfl$.

      \begin{graycase}
        \begin{mathc}
          \ru{\Gamma \derr F_1 = F_2 : \Pi x:A.B \qquad
              \Gamma \derr a_1 = a_2 : A
            }{\Gamma \derr F_1\ a_1 = F_2\ a_2 : B[x := a_1]}
        \end{mathc}
        By induction hypotheses,
        $\Gamma' \dere h_1 : \Id (\Pi x:A'.B')\ F'_1\ F'_2$ and
        $\Gamma' \dere h_2 : \Id A'\ a'_1\ a'_2$ (here we use an elaborate
        version of the transport trick since the original translation gives us
        $A'$ and $B'$).
        The only problem here is that $F'_2\ a'_2$ would have type
        $B'[x := a'_2]$ instead of $B'[x := a'_1]$ but these two types are equal
        (as in there exists an identity proof) so we can conclude with a
        transport around $F'_2\ a'_2$, but we won't state anything explicitely.
      \end{graycase}

      \begin{graycase}
        \begin{mathc}
          \ru{\Gamma \derr A_1 = A_2 : s \qquad
              \Gamma \derr u_1 = u_2 : A_1 \qquad
              \Gamma \derr v_1 = v_2 : A_1
            }{\Gamma \derr \Id A_1\ u_1\ v_1 = \Id A_2\ u_2\ v_2 : s}
        \end{mathc}
        By induction hypotheses, $\Gamma' \dere h_A : \Id s\ A'_1\ A'_2$
        and $\Gamma' \dere h_u : \Id A'_1\ u'_1\ u'_2$ and
        $\Gamma' \dere h_v : \Id A'_1\ v'_1\ v'_2$.
        Destructing these three equalities \emph{is all it takes\footnote{Thus
        avoiding to do it \emph{in extenso} here.}} to prove
        $\Gamma' \dere h : \Id s\ (\Id A'_1\ u'_1\ v'_1)\ %
        (\Id A'_2\ ({h_A}_*\ u'_2)\ ({h_A}_*\ v'_2))$ for some $h$.
      \end{graycase}

      \begin{graycase}
        \begin{mathc}
          \ru{\Gamma \derr A_1 = A_2 : s \qquad
              \Gamma \derr u_1 = u_2 : A_1
            }{\Gamma \derr \refl(A_1,u_1) = \refl(A_2,u_2) : \Id A_1\ u_1\ u_1}
        \end{mathc}
        By induction hypotheses, $\Gamma' \dere h_A : \Id s\ A'_1\ A'_2$
        and $\Gamma' \dere h_u : \Id A'_1\ u'_1\ u'_2$.
        This case is treated as the previous one by saying there is an $h$
        such that $\Gamma' \dere h : \Id (\Id A'_1\ u'_1\ u'_1)\ %
        \refl(A'_1,u'_1)\ (h'_*\ \refl(A'_2,u'_2))$ where $h'$ is an equality
        like the one from the precedent case.
      \end{graycase}

      \begin{graycase}
        \begin{mathc}
          \tru{\Gamma \derr A_1 = A_2 : s \qquad
               \Gamma \derr C_1 = C_2 : \Pi x:A_1. \Pi y:A_1. (\Id A_1\ x\ y)
               \to s'
             }{\Gamma \derr b_1 = b_2 : \Pi x:A_1. C_1\ x\ x\ \refl(A_1,x)
             }{\Gamma \derr u_1 = u_2 : A_1 \qquad
               \Gamma \derr v_1 = v_2 : A_1 \qquad
               \Gamma \derr p_1 = p_2 : \Id A_1\ u_1\ v_1
             }{\Gamma \derr \J(A_1,C_1,b_1,u_1,v_1,p_1) =
               \J(A_2,C_2,b_2,u_2,v_2,p_2) :
               C_1\ u_1\ v_1\ p_1}
        \end{mathc}
        This is again treated the same way, with a transport on the right to
        make everything
        type\footnote{I don't think anyone would read it anyway.}.
      \end{graycase}

      \begin{graycase}
        \begin{mathc}
          \dru{\Gamma \derr A : s \qquad
               \Gamma \derr C : \Pi x:A. \Pi y:A. (\Id A\ x\ y) \to s'
             }{\Gamma \derr b : \Pi x:A. C\ x\ x\ \refl(A,x) \qquad
               \Gamma \derr u : A
             }{\Gamma \derr \J (A,C,b,u,u,\refl(A,u)) = b\ u :
               C\ u\ u\ \refl(A,u)}
        \end{mathc}
        This time, it's basically like $\beta$-reduction, we use $\refl$.
      \end{graycase}

      \begin{graycase}
        \begin{mathc}
          \ru{\Gamma \derr a_1 = a_2 : A_1 \qquad
              \Gamma \derr A_1 = A_2 : s
            }{\Gamma \derr a_1 = a_2 : A_2}
        \end{mathc}
        By induction hypotheses, $\Gamma' \dere h_a : \Id A'_1\ a'_1\ a'_2$
        and $\Gamma' \dere h_A : \Id s\ A'_1\ A'_2$.
        We easily get $h$ such that $\Gamma' \dere h : \Id A'_2\ %
        ({h_A}_*\ a'_1)\ ({h_A}_*\ a'_2)$.
      \end{graycase}

      \nextcase
      \begin{mathc}
        \ru{\derr \Gamma
          }{\Gamma \derr \mR : \sB}
      \end{mathc}
      By induction hypothesis, there exists $\Gamma'$ a well-formed translation
      of $\Gamma$. And for any such translation, $\Gamma' \dere \mB : \sB$
      which is a valid translation.

      \nextcase
      \begin{mathc}
        \ru{\Gamma \derr t : \mA
          }{\Gamma \derr \inj(t) : \mR}
      \end{mathc}
      By induction hypothesis, we have $\Gamma' \dere t' : A'$.
      Since $\mA$ and $A'$ can be equated (lemma~\ref{lem:transleq}
      \meta{plus a lemma that says that any term of $\Se$ is a translation of
      itself?}),
      we have a transport $h_* : A' \to \mA$, so
      $\Gamma' \dere h_*\ t' : \mA$ and thus
      $\Gamma' \dere \mbe_*\ (h_*\ t') : \mbB$ which gives us a desired
      translation... \meta{This actually doesn't work, we have to figure it out!
      If we want to use $\mbe$, we need to start from $\mbA$.
      While it makes sense with $\pfull$, what does it mean under $\prfl$?}

      \nextcase
      \begin{mathc}
        \ru{\Gamma \derr t : \mR
          }{\Gamma \derr \proj(t) : \mA}
      \end{mathc}
      By induction hypothesis, $\Gamma' \dere t' : R'$.
      We can safely assume $\Gamma' \dere t' : \mB$ and thus
      $\Gamma' \dere {\me\inv}_*\ t' : \mA$.
      \meta{Same problem...}

      \nextcase
      \begin{mathc}
        \ru{\Gamma \derr t : \mA
          }{\Gamma \derr \proj(\inj(t)) = t : \mA}
      \end{mathc}
      By induction hypothesis, $\Gamma' \dere t' : \mA$.
      Then $\eta\ t'$ is the proof we seek, where $\eta^\bullet$
      is $\eta$ using the variables $\mbA,\mbB,\mbe$ instead of
      $\mA,\mB,\me$ (it indeed does not need to know anything about them).

      \nextcase
      \begin{mathc}
        \ru{\Gamma \derr t : \mR
          }{\Gamma \derr \inj(\proj(t)) = t : \mR}
      \end{mathc}
      By induction hypothesis, $\Gamma' \dere t' : \mB$.
      Then $\epsilon^\bullet\ t'$ is the proof we seek, where $\epsilon^\bullet$
      is $\epsilon$ using the variables $\mbA,\mbB,\mbe$ (like $\eta$).
    \end{caselist}
  \end{proof}

  \begin{corollary}[Consistency]
    \label{cor:cons1}
    The least type is not inhabited: $X : \Type_0 \nvdash_r t : X$
    (as long as it is not inhabited in $\Se$).
  \end{corollary}

  \begin{proof}
    Let's write $\Gamma := X : \Type_0$ and assume $\Gamma \derr t : X$.
    Thus, by theorem~\ref{thm:transl}, we have $\Gamma \dere t' : T$ with
    the right translations (we indeed can choose whatever we want for the
    translation of $\Gamma$ and $\Gamma$ itself is valid).
    \Wlog, by lemma~\ref{lem:transleq}, we can assume
    $\Gamma \dere t' : X$ (by applying a transport to $t'$).
    Now, assuming that $\Se$ is itself consistent, this is a contradiction.
    Thus, our system with resizing rules is consistent (with respect to $\Se$).
  \end{proof}

  \section{Consistency of Other Resizing Rules}

  Now that we treated the case of the resizing of equality by adding a
  pseudo-record $\mR$ representing the resized $\mA$ (\ie in the universe of
  $\mB$), we can handle different kinds of resizing rules as long as we assume
  some axioms.
  Please note that this is only to prove consistency, the user doesn't have to
  rely on these extra axioms.

  \paradot{Resizing of Equivalence}

  Assuming univalence, this one is pretty straightforward.
  Indeed, if two type are equivalent,
  they can be stated equal as a result of univalence. Instead of resizing an
  equivalence we are then resizing equality.

  \paradot{Resizing of Propositions}

  If we assume the law of excluded-middle, it is possible to create an
  equivalence between any mere proposition and either the empty type or the
  unit type (depending on the result of the excluded middle on this
  proposition). The only natural way of defining such maps yields an equivalence
  (coherences hold because of the proposition property or by \emph{exfalso}).

  Since, both the empty type and the unit type inhabit the smallest universe,
  by using the resizing for equivalence, we know that any mere proposition lives
  in the smallest universe (or equivalently any universe).

  In a similar fashion, one can resize the type of mere propositions to the
  smallest universe\footnote{Or any universe that shelters the booleans.} as
  well by remarking that it can be shown equivalent to
  the type of booleans in presence of the excluded middle.

  \paradot{Resizing of Embeddings}

  If we have an embedding $f$ of $A$ into $B$, then $A$ can be shown equivalent
  to its image through $f$ : $\{ b : B \mid \| \exists a:A. b = f(a) \| \}$.
  Since the squash is a mere proposition, the image can be resized to the
  universe of $B$ and so can $A$.

  Note that this rule entails the one on equivalence (and thus equality) and
  the one on propositions (but not the resizing of the type of propositions
  itself).

\end{document}
