\documentclass[a4paper,english]{lipics-utf8x}

\usepackage[T1]{fontenc} %

\usepackage{amsmath, amssymb, amsfonts, stmaryrd}
\usepackage{pifont}
\PrerenderUnicode{é} % For the author names in the heading

% Add some colors
\usepackage[usenames,dvipsnames,svgnames,table]{xcolor}
\usepackage{hyperref}
\hypersetup{
 linktocpage,
 colorlinks,
 citecolor=BlueViolet,
 filecolor=red,
 linkcolor=Blue,
 urlcolor=BrickRed
}

\usepackage{graphicx}
\usepackage{placeins}

% Meta comment
\newcommand\meta[1]{\noindent\textcolor{blue}{\emph{#1}}}

% Include the macro file
% evergreens
\newcommand{\der}{\,\vdash}
\newcommand{\Der}{\,\Vdash}

% semantic brackets
\def\lv{\mathopen{{[\kern-0.14em[}}}    % opening [[ value delimiter
\def\rv{\mathclose{{]\kern-0.14em]}}}   % closing ]] value delimiter
\newcommand{\den}[1]{\lv #1 \rv}
\newcommand{\Den}[3][]{\den{#2}^{#1}_{#3}}
\newcommand{\dent}[2]{\llparenthesis#1\rrparenthesis_{#2}}

% latin etc. abbrev
\newcommand{\abbrev}[1]{#1} % alternative: \emph{#1}
\newcommand{\cf}{\abbrev{cf.}\ }
\newcommand{\eg}{\abbrev{e.\,g.}}
\newcommand{\Eg}{\abbrev{E.\,g.}}
\newcommand{\ie}{\abbrev{i.\,e.}}
\newcommand{\Ie}{\abbrev{I.\,e.}}
\newcommand{\etal}{\abbrev{et.\,al.}}
\newcommand{\wwlog}{w.\,l.\,o.\,g.} % \wlog is ``write into log file''
\newcommand{\Wlog}{W.\,l.\,o.\,g.}
\newcommand{\wrt}{w.\,r.\,t.}

% Inference rules
\newcommand{\rulename}[1]{\ensuremath{\mbox{\sc#1}}}
\newcommand{\ru}[2]{\dfrac{\begin{array}[b]{@{}c@{}} #1 \end{array}}{#2}}
\newcommand{\rux}[3]{\ru{#1}{#2}~#3}
\newcommand{\nru}[3]{#1\ \ru{#2}{#3}}
\newcommand{\nrux}[4]{#1\ \ru{#2}{#3}\ #4}
\newcommand{\dstack}[2]{\begin{array}[b]{c}#1\\#2\end{array}}
\newcommand{\dru}[3]{\ru{\dstack{#1}{#2}}{#3}}
\newcommand{\tru}[4]{\dru{\dstack{#1}{#2}}{#3}{#4}}
\newcommand{\trux}[5]{\dru{\dstack{#1}{#2}}{#3}{#4}\ #5}
\newcommand{\qru}[5]{\tru{\dstack{#1}{#2}}{#3}{#4}{#5}}
\newcommand{\ndru}[4]{#1\ \ru{\dstack{#2}{#3}}{#4}}
\newcommand{\ndrux}[5]{#1\ \ru{\dstack{#2}{#3}}{#4}\ #5}

% Symbols and names
\newcommand\Type{\operatorname{Type}}
\newcommand\isnType[2]{\operatorname{is-}#1\operatorname{-Type}\ #2}
\newcommand\nType[1]{#1\operatorname{-Type}}
\newcommand\R{\operatorname{R}}
\newcommand\emb[2]{\operatorname{embedding}#1\ #2}
\newcommand\RRe[2]{\operatorname{RR_e}#1\ #2}
% \newcommand\type{\ \bm{\operatorname{type}}}
\DeclareMathOperator{\type}{\ \mathbf{type}}
\DeclareMathOperator{\ctr}{\mathbf{ctr}}
\DeclareMathOperator{\refl}{\mathbf{refl}}
\newcommand\rew{\searrow}
\newcommand\gettype{\operatorname{.type}}
\newcommand\getproof{\operatorname{.proof}}
\newcommand\Var{\operatorname{Var}}
\newcommand\Exp{\operatorname{Exp}}
\newcommand\Ctx{\operatorname{Ctx}}
\newcommand\Whnf{\operatorname{Whnf}}
\newcommand\Wne{\operatorname{Wne}}


% Title and so...
\title{Working Resizing of Equivalence under Univalence}
\author[1]{Théo Winterhalter}

\begin{document}

  \maketitle

  \begin{abstract}
    ...
  \end{abstract}

  \section{Our systems}

  This is the syntax of the system in which we start.

  \[
    \begin{array}{l@{~}l@{~}l@{~}r@{~}l@{\quad}l}
      \Var  & \ni & x,y,X,Y \\
      \Sort & \ni & s             & ::= & \Type_k \mbox{ }
                                                (k \in \mathbb{N}) \\
      \Exp  & \ni & t,u,T,U & ::= & s \mid \Pi x:U.T \mid
                                    \Id T\ t\ u \\
                         &&& \mid & x \mid \lambda x:U.t \mid t~u
                               \mid \refl(T,t) \mid
                               \J (T,U,t_{refl},u_1,u_2,t_{eq}) \\
      \Ctx  & \ni & \Gamma  & ::= & \cdot \mid \Gamma, x:T \\
    \end{array}
  \]
  %
  We assume it contains $\mA$ and $\mB$ that are equivalent (and respectively in
  universes $\sA$ and $\sB$), but up to univalence, we only assume that
  there is $e$ such that $\der e : \Id s\ \mA\ \mB$ for some $s$
  (note: we will only treat the case where this holds in the empty context,
  but it should scale easily).

  We then extend this system with $\mR$, $\inj(t)$ and $\proj(t)$ as well as
  the following rules (which we will note with $\derr$ instead of $\dere$).

  \begin{mathc}
    \ru{\derr \Gamma
      }{\Gamma \derr \mR : \sB}
    \qquad
    \ru{\Gamma \derr t : \mA
      }{\Gamma \derr \proj(\inj(t)) = t : \mA}
    \qquad
    \ru{\Gamma \derr t : \mR
      }{\Gamma \derr \inj(\proj(t)) = t : \mR}
  \end{mathc}

  \section{Translation for Consistency}

  \begin{theorem}[Translation]
    \leavevmode
    \begin{enumerate}
      \item If $\Gamma \derr t : T$ then there exists $\Gamma'$ a well-formed
      translation of $\Gamma$ and for any such $\Gamma'$, there are $t'$ and
      $T'$ translations of $t$, $T$ such that $\Gamma' \dere t' : T'$.
      \item If $\Gamma \derr t = u : T$ then there exists $\Gamma'$ a
      well-formed translation of $\Gamma$ and for any such $\Gamma'$, there are
      $t'$, $u'$ and $T'$ translations of $t$, $u$, $T$ and $h$ such that
      $\Gamma' \dere h : \Id T'\ t'\ u'$.
      \item If $\derr \Gamma$ then there exists $\Gamma'$ a translation of
      $\Gamma$ such that $\dere \Gamma'$.
    \end{enumerate}
  \end{theorem}

  \meta{We have to define what \emph{translation} means}.

  \begin{proof}
    By induction on the derivation.

    \leavevmode
    \begin{caselist}
      \nextcase
      \begin{mathc}
        \ru{\Gamma \derr a : A \qquad
            \Gamma \derr A = B : s
          }{\Gamma \derr a : B}
      \end{mathc}
      By the first induction hypothesis, there is $\Gamma'$, $a'$ and $A'$
      such that $\Gamma' \dere a' : A'$ (and that are translations).
      Then, we can use the second induction hypothesis with $\Gamma'$.
      \meta{We would also need to be able to provide $A'$ or make sure that
      two translations of the same term are equal (only under the same type
      obviously).}

      \nextcase
      \begin{mathc}
        \rux{\Gamma \derr A_1 = A_2 : s_1 \qquad
             \Gamma, x : A_1 \derr b_1 = b_2 : B : s_2
           }{\Gamma \derr \lambda x:A_1.b_1 = \lambda x:A_2.b_2 : \Pi x:A_1.B
           }{(s_1,s_2,s_3)}
      \end{mathc}
      \meta{It seems that if they give me a translation of $\lambda x:A_1.b_1$,
      I cannot deduce one of $A_1$ (at least not that easily).}
    \end{caselist}
  \end{proof}

\end{document}
