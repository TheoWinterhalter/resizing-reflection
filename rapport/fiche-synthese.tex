\documentclass[11pt]{article}

\usepackage[francais]{babel} %
\usepackage[T1]{fontenc} %
% \usepackage[latin1]{inputenc} %
\usepackage[utf8]{inputenc} %
% \usepackage[applemac]{inputenc} %
% \usepackage{a4wide} %

% \setlength{\parskip}{0.3\baselineskip}

\usepackage{amsmath, amssymb, amsfonts, stmaryrd}
\usepackage{bm}
\usepackage{pifont}

% Add some colors
\usepackage[usenames,dvipsnames,svgnames,table]{xcolor}
\usepackage{hyperref}
\hypersetup{
 linktocpage,
 colorlinks,
 citecolor=BlueViolet,
 filecolor=red,
 linkcolor=Blue,
 urlcolor=BrickRed
}

% Meta comment
\newcommand\meta[1]{\noindent\textcolor{blue}{\emph{#1}}}

\begin{document}

\title{Vers une Justification des \emph{Resizing Rules} en Théorie Homotopique
       des Types}

\author{Théo Winterhalter, Nicolas Tabareau (ASCOLA, LINA)}

\date{5 Septembre 2016}

\maketitle

\pagestyle{empty} %
\thispagestyle{empty}

%% Attention: pas plus d'un recto-verso!
% Ne conservez pas les questions


\subsection*{Le contexte général}

% De quoi s'agit-il ?
% D'où vient-il ?
% Quels sont les travaux déjà accomplis dans ce domaine dans le monde ?
\sloppy %
La théorie homotopique des types (\emph{Homotopy Type Theory}, abrégée HoTT)
est un domaine récent qui propose une nouvelle axiomatisation des mathématiques
notamment en proposant des fondations univalentes. L'univalence consiste à
formaliser un raisonnement qui est souvent fait implicitement en prose
mathématique consitstant à considérer indifféremment deux structures
isomorphes : autrement dit à les considérer égales.
La théorie sous-jacente élabore donc de nombreux résultats liés à l'égalité
(ainsi que les cohérences supérieures, égalités entre égalités) qui diffèrent
des axiomatisatons usuelles qui souvent implicitement unifiaient les différentes
manières de prouver deux objets égaux.

\subsection*{Le problème étudié}

% Quelle est la question que vous avez abordée ?
% Pourquoi est-elle importante, à quoi cela sert-il d'y répondre ?
% Est-ce un nouveau problème ?
% Si oui, pourquoi êtes-vous le premier chercheur de l'univers à l'avoir posée ?
% Si non, pourquoi pensiez-vous pouvoir apporter une contribution originale ?
En théorie des types, et donc dans des assistants de preuve tels que Coq ou
Agda, pour éviter des paradoxes à la Russel, on établit une hiérarchie d'univers
(les types des types) assurant alors la cohérence du système (sans ça, il serait
possible de prouver \emph{faux} en Coq par exemple).
Néanmoins, les règles concernant les univers sont syntaxiques et en quelque
sorte une surapproximation. Il est donc raisonnable de se demander si dans
certain cas il est possible de les assouplir.
Énoncées par Vladimir Voevodsky, les \emph{resizing rules} sont des règles
qui permettent d'abaisser le niveau d'univers d'un type suivant des conditions
précises. La question est donc de comment les justifier et de comment les
instrumenter pour qu'elles soient utilisables en pratique dans Coq par exemple.

\subsection*{La contribution proposée}

% Qu'avez vous proposé comme solution à cette question ?
% Attention, pas de technique, seulement les grandes idées !
% Soignez particulièrement la description de la démarche \emph{scientifique}.
Ma contribution consiste en une proposition d'implémentation dans Coq
(qui s'accompagne par une réelle implémentation dans Coq par l'un des membres
de l'équipe) ainsi que des pistes pour une justification qui soit compatible
avec l'axiome d'univalence. En plus de proposer une nouvelle \emph{resizing
rule}, on montre que l'on peut ramener toutes les règles à une seule (à
condition de s'autoriser le tiers-exclu pour les propositions).

\subsection*{Les arguments en faveur de sa validité}

% Qu'est-ce qui montre que cette solution est une bonne solution ?
% Des expériences, des corollaires ?
% Commentez la \emph{robustesse} de votre proposition :
% comment la validité de la solution dépend-elle des hypothèses de travail ?
Nous n'avons malheureusement pas été en mesure d'établir la cohérence de nos
\emph{resizing rules} néanmoins on propose une ébauche de preuve où toutes les
étapes sont montrées exceptée une étape qui repose sur une conjecture.

\subsection*{Le bilan et les perspectives}

Et après ? En quoi votre approche est-elle générale ?
Qu'est-ce que votre contribution a apporté au domaine ?
Que faudrait-il faire maintenant ?
Quelle est la bonne \emph{prochaine} question ?

\newpage

\section{Théorie Homotopique des Types}

\subsection{Théorie des types et assistants de preuve}

\meta{À moins de s'en servir comme introduction plutôt.}

\subsection{L'égalité au cœur des preuves}

\subsection{Principe d'univalence}

\section{\emph{Resizing Rules}}

\subsection{Introduction / Justification}

\subsection{Enoncé Originel}

\subsection{Notre proposition d'implémentation}

\section{Vers une démonstration}

\subsection{Une règle pour les gouverner toutes}

\subsection{Traduction et différentes pistes}

\section{Conclusion}

\end{document}
