\documentclass[11pt]{article}

\usepackage[francais]{babel} %
\usepackage[T1]{fontenc} %
% \usepackage[latin1]{inputenc} %
\usepackage[utf8]{inputenc} %
% \usepackage[applemac]{inputenc} %
% \usepackage{a4wide} %

% \setlength{\parskip}{0.3\baselineskip}

\begin{document}

\title{Vers une Justification des \emph{Resizing Rules} en Théorie Homotopique
       des Types}

\author{Théo Winterhalter, Nicolas Tabareau (ASCOLA, LINA)}

\date{5 Septembre 2016}

\maketitle

\pagestyle{empty} %
\thispagestyle{empty}

%% Attention: pas plus d'un recto-verso!
% Ne conservez pas les questions


\subsection*{Le contexte général}

% De quoi s'agit-il ?
% D'où vient-il ?
% Quels sont les travaux déjà accomplis dans ce domaine dans le monde ?
\sloppy %
La théorie homotopique des types (\emph{Homotopy Type Theory}, abrégée HoTT)
est un domaine récent qui propose une nouvelle axiomatisation des mathématiques
notamment en proposant des fondations univalentes. L'univalence consiste à
formaliser un raisonnement qui est souvent fait implicitement en prose
mathématique consitstant à considérer indifféremment deux structures
isomorphes : autrement dit à les considérer égales.
La théorie sous-jacente élabore donc de nombreux résultats liés à l'égalité
(ainsi que les cohérences supérieures, égalités entre égalités) qui diffèrent
des axiomatisatons usuelles qui souvent implicitement unifiaient les différentes
manières de prouver deux objets égaux.

\subsection*{Le problème étudié}

Quelle est la question que vous avez abordée ?
Pourquoi est-elle importante, à quoi cela sert-il d'y répondre ?
Est-ce un nouveau problème ?
Si oui, pourquoi êtes-vous le premier chercheur de l'univers à l'avoir posée ?
Si non, pourquoi pensiez-vous pouvoir apporter une contribution originale ?

\subsection*{La contribution proposée}

Qu'avez vous proposé comme solution à cette question ?
Attention, pas de technique, seulement les grandes idées !
Soignez particulièrement la description de la démarche \emph{scientifique}.

\subsection*{Les arguments en faveur de sa validité}

Qu'est-ce qui montre que cette solution est une bonne solution ?
Des expériences, des corollaires ?
Commentez la \emph{robustesse} de votre proposition :
comment la validité de la solution dépend-elle des hypothèses de travail ?

\subsection*{Le bilan et les perspectives}

Et après ? En quoi votre approche est-elle générale ?
Qu'est-ce que votre contribution a apporté au domaine ?
Que faudrait-il faire maintenant ?
Quelle est la bonne \emph{prochaine} question ?

\end{document}
