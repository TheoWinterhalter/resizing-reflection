\documentclass[11pt]{article}

% Font
% \renewcommand{\familydefault}{\sfdefault}
% \usepackage{helvet}
\usepackage[sfdefault,light]{roboto}

\usepackage[T1]{fontenc} %
% \usepackage[latin1]{inputenc} %
\usepackage[utf8]{inputenc} %
% \usepackage[applemac]{inputenc} %
% \usepackage{a4wide} %

% \setlength{\parskip}{0.3\baselineskip}

\usepackage{amsmath, amssymb, amsfonts, stmaryrd}
\usepackage{bm}
\usepackage{pifont}

% Add some colors
\usepackage[usenames,dvipsnames,svgnames,table]{xcolor}
\usepackage{hyperref}
\hypersetup{
 linktocpage,
 colorlinks,
 citecolor=BlueViolet,
 filecolor=red,
 linkcolor=Blue,
 urlcolor=BrickRed
}

% NTH
\usepackage[super]{nth}

% Meta comment
\newcommand\meta[1]{\noindent\textcolor{blue}{\emph{#1}}}

\begin{document}

\title{Justifying Resizing Rules in HoTT \\
\small{Through an interpretation of extensional type theory into an intensional
one}}

\author{Théo Winterhalter, Nicolas Tabareau (ASCOLA, LINA)}

\date{\nth{5} of September, 2016}

\maketitle

\pagestyle{empty} %
\thispagestyle{empty}

%% Attention: pas plus d'un recto-verso!
% Ne conservez pas les questions


\subsection*{The general context}
% What is it about ? Where does it come from ?
% What is the state of the art in this area ?
Homotopy Type Theory (HoTT for short) is a new domain that aims to offer a new
axiomatisation of mathematics that allows to consider objects up to isomorphisms
(as a result of the univalence axiom which allows to derive an equality between
isomorphic types).
The underlying theory thus develops a lot of notions related to equality and its
treatment, in particular it theorises about equalities between equalities that
are often considered to be trivial as they are in Set Theory.

\subsection*{The research problem}

% What is the question that you studied ?
% Why is it important, what are the applications/consequences ?
% Is it a new problem ?
% If so, why are you the first researcher in the universe who consider it ?
% If not, why did you think that you could bring an original contribution ?
In type theory, and thus in proof assistants such as Coq or Agda, in order to
avoid paradoxes \emph{à la} Russel, we have to establish a hierarchy of
universes (the types of types). Without it, it would be possible to prove
\emph{false} in Coq for instance.
Nevertheless, the typing rules dealing with universes are purely syntactical
and can be seen as rough approximations to preserve consistency.
They can be annoying for the user and we can legitimately wonder whether we
can alleviate theses restrictions in some specific cases.

Stated by Vladimir Voevodsky, the \emph{resizing rules} allow to lower the
universe level of a type under special conditions. The way he presented them
would make type checking undecidable.
We studied the possibility of a definition better suited to implementation
as well as a justification that the rules we propose preserve the consistency
of the system.

\subsection*{Your contribution}

% What is your solution to the question described in the last paragraph ?
% Be careful, do \emph{not} give technical details, only rough ideas !
% Pay a special attention to the description  of the \emph{scientific} approach.
We propose a way to implement resizing rules in Coq (which will be possible in
the next release) with precise typing rules.
We then show that -- up to some axioms that are consistent in a univalent
setting -- all resizing rules that we proposed (and that correspond to the ones
of Voevodsky) can be reduced to only one.
We finally show the consistency of this one rule by generalizing a result by
Marc Oury stating that we can translate extensional type theory into
intensional type theory in a univalent setting (up to now it relied on the
uniqueness of identity proofs that is incompatible with univalence).
This last step is, in itself, an interesting result.

\subsection*{Arguments supporting its validity}

% What is the evidence that your solution is a good solution ?
% Experiments ? Proofs ?
Our work consists of a proof and an implementation.
However, our proof relies on some axioms that we deem consistent in a univalent
setting but this would require more work to justify their use more explicitely.

% Comment the robustness of your solution: how does it rely/depend on the working assumptions ?

\subsection*{Summary and future work}

% What is next ? In which respect is your approach general ?
% What did your contribution bring to the area ?
% What should be done now ?
% What is the good \emph{next} question ?
Our approach goes through a very gereral interpretation of extensional type
theory into intensional type theory which is an interesting result unavailable
in homotopy type theory. Besides we allow the use of resizing rules in actual
proofs in Coq.
Still, as we pointed out, we rely on an axiom and it would be interesting to
see if it holds in the simplicial model which is the mainstream model of
univalent homotopy type theory.

\subsection*{Notes}

Given that the translation from extensional type theory into intensional type
theory without using uniqueness of identity proofs and that justifying
resizing rule might serve as a basis for articles we decided to write this
report in English.

\newpage

\section{Homotopy Type Theory}

\subsection{Proof Theory and Proof Assistants}

\subsection{Proofs and Equality}

\subsection{Univalence}

\section{Resizing Rules}

\subsection{Introduction / Justification}

\subsection{Original Statement}

\subsection{Implementation Proposal}

\section{One Rule to Resize Them All}

\section{Extensional to Intensional}

\section*{Conclusion}

\end{document}
